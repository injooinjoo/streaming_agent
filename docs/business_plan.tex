% =============================================================================
% Streaming Agent 사업계획서
% 오버레이 기반 스트리밍 데이터 수집 플랫폼
% =============================================================================
\documentclass[11pt, a4paper]{report}

% --- 한국어 지원 ---
\usepackage{fontspec}
\usepackage{polyglossia}
\setdefaultlanguage{korean}
\setotherlanguage{english}
\setmainfont{Apple SD Gothic Neo}
\setsansfont{Apple SD Gothic Neo}
\setmonofont{Menlo}
\newfontfamily\hangulfonttt{Apple SD Gothic Neo}

% --- 레이아웃 ---
\usepackage[top=2.5cm, bottom=2.5cm, left=2.5cm, right=2.5cm]{geometry}
\usepackage{setspace}
\onehalfspacing
\usepackage{fancyhdr}
\pagestyle{fancy}
\fancyhf{}
\fancyhead[L]{\small\leftmark}
\fancyhead[R]{\small Streaming Agent 사업계획서}
\fancyfoot[C]{\thepage}
\renewcommand{\headrulewidth}{0.4pt}

% --- 표 ---
\usepackage{booktabs}
\usepackage{longtable}
\usepackage{tabularx}
\usepackage{array}
\newcolumntype{L}[1]{>{\raggedright\arraybackslash}p{#1}}
\newcolumntype{C}[1]{>{\centering\arraybackslash}p{#1}}

% --- 그래픽 ---
\usepackage{graphicx}
\usepackage{float}
\usepackage{tikz}
\usetikzlibrary{shapes.geometric, arrows.meta, positioning, fit, calc, backgrounds, decorations.pathreplacing}

% --- 코드 ---
\usepackage{listings}
\usepackage{xcolor}

\definecolor{codebg}{RGB}{245,245,250}
\definecolor{codeframe}{RGB}{200,200,220}
\definecolor{keyword}{RGB}{0,100,200}
\definecolor{string}{RGB}{0,128,0}
\definecolor{comment}{RGB}{128,128,128}

\lstdefinestyle{sqlstyle}{
  backgroundcolor=\color{codebg},
  frame=single,
  rulecolor=\color{codeframe},
  basicstyle=\ttfamily\scriptsize,
  keywordstyle=\color{keyword}\bfseries,
  stringstyle=\color{string},
  commentstyle=\color{comment}\itshape,
  language=SQL,
  breaklines=true,
  showstringspaces=false,
  tabsize=2
}

\lstdefinestyle{jsstyle}{
  backgroundcolor=\color{codebg},
  frame=single,
  rulecolor=\color{codeframe},
  basicstyle=\ttfamily\scriptsize,
  keywordstyle=\color{keyword}\bfseries,
  stringstyle=\color{string},
  commentstyle=\color{comment}\itshape,
  language=Java,
  morekeywords={const,let,async,await,function,return,export,import,from},
  breaklines=true,
  showstringspaces=false,
  tabsize=2
}

% --- 색상 정의 ---
\definecolor{nexonblue}{RGB}{30,39,97}
\definecolor{accentgreen}{RGB}{0,200,150}
\definecolor{soop}{RGB}{89,182,255}
\definecolor{chzzk}{RGB}{0,255,163}
\definecolor{youtube}{RGB}{255,0,0}
\definecolor{twitch}{RGB}{145,70,255}
\definecolor{warnred}{RGB}{220,50,50}
\definecolor{okgreen}{RGB}{50,180,80}

% --- 하이퍼링크 ---
\usepackage{hyperref}
\hypersetup{
  colorlinks=true,
  linkcolor=nexonblue,
  citecolor=nexonblue,
  urlcolor=nexonblue
}

% --- 기타 ---
% enumitem 제거 - 기본 LaTeX 리스트 사용
\usepackage{caption}
\usepackage{subcaption}

% --- TikZ 스타일 ---
\tikzset{
  box/.style={draw, rounded corners=3pt, minimum height=1cm, minimum width=2.5cm, align=center, font=\small},
  platform/.style={box, fill=blue!10, draw=blue!50},
  adapter/.style={box, fill=orange!10, draw=orange!50},
  process/.style={box, fill=green!10, draw=green!50},
  storage/.style={box, fill=purple!10, draw=purple!50},
  client/.style={box, fill=yellow!10, draw=yellow!50},
  layer/.style={draw, dashed, rounded corners=5pt, inner sep=8pt, fill opacity=0.05},
  arrow/.style={-{Stealth[length=3mm]}, thick},
  biarrow/.style={{Stealth[length=3mm]}-{Stealth[length=3mm]}, thick},
}

% =============================================================================
\begin{document}

% =============================================================================
% 표지
% =============================================================================
\begin{titlepage}
\begin{center}
\vspace*{3cm}

{\Huge\bfseries Streaming Agent}\\[0.8cm]
{\LARGE 오버레이 기반 스트리밍 데이터 수집 플랫폼}\\[0.3cm]
{\LARGE 사업 계획서}\\[2cm]

{\large\color{gray} 플랫폼 의존에서 벗어나\\[0.2cm]
게임 중심의 스트리밍 인사이트를 확보하는 방법}\\[3cm]

\rule{0.6\textwidth}{0.5pt}\\[1cm]

{\large 넥슨 크리에이터파트너십팀}\\[0.3cm]
{\large 김인주}\\[0.5cm]
{\large\color{gray} 내부 전략 제안서}\\[1cm]
{\large 2026년 1월}\\[2cm]

{\small\color{gray} CONFIDENTIAL --- 본 문서는 내부 검토용이며 외부 배포를 금합니다.}

\end{center}
\end{titlepage}

% =============================================================================
% Executive Summary
% =============================================================================
\chapter*{Executive Summary}
\addcontentsline{toc}{chapter}{Executive Summary}

\begin{center}
\begin{tikzpicture}
  \node[box, fill=nexonblue!10, draw=nexonblue, minimum width=4.5cm, minimum height=1.5cm] at (0,0) {\textbf{데이터 독점 확보}\\전 플랫폼 실시간 수집};
  \node[box, fill=nexonblue!10, draw=nexonblue, minimum width=4.5cm, minimum height=1.5cm] at (5.5,0) {\textbf{ROI 측정 가능}\\캠페인 성과 정량화};
  \node[box, fill=nexonblue!10, draw=nexonblue, minimum width=4.5cm, minimum height=1.5cm] at (11,0) {\textbf{생태계 지배력}\\스트리밍 인프라 구축};
\end{tikzpicture}
\end{center}

\vspace{0.5cm}

\textbf{Streaming Agent}는 스트리머 방송 화면 위에서 동작하는 오버레이 기반 에이전트로, 멀티 플랫폼(SOOP, 치지직, YouTube, Twitch)의 방송·채팅·후원·이벤트 데이터를 실시간으로 수집·정규화·분석하는 데이터 인입 시스템이다.

현재 스트리밍은 게임 마케팅에서 매우 중요한 채널임에도 불구하고, 데이터는 플랫폼에 종속되어 있으며, 스트리머 운영과 캠페인 성과 분석은 감각과 경험 중심으로 이루어지고 있다. 플랫폼은 법률적·사업전략적 이유로 타 게임, 타 카테고리 데이터, 후원금 정보를 제공하지 않는다.

본 사업은 플랫폼과 경쟁하지 않는다. \textbf{모든 이벤트가 통과하는 '툴(오버레이)' 레이어를 확보함으로써 데이터 주도권을 획득}하고, 스트리머--유저--게임을 하나의 데이터 루프로 연결하는 것을 목표로 한다.

\vspace{0.3cm}
\noindent\textbf{핵심 전략:}
\begin{enumerate}
  \item 모든 스트리머가 쓰는 오버레이를 만들어 \textbf{결집}시킨다
  \item 그 구조 위에서 넥슨이 \textbf{데이터를 수집할 수 있는 파이프라인}을 확보한다
\end{enumerate}

\noindent\textbf{현재 시스템 현황:} 35개 데이터베이스 테이블, 50+ REST API 엔드포인트, 10종 오버레이, 실시간 Socket.io 이벤트 파이프라인 구축 완료.

\vspace{0.3cm}
\noindent\textbf{사업 모델:} 모든 기능을 무료로 제공하여 스트리머 기반을 빠르게 확보하고, 오버레이 내 광고 인벤토리를 통한 수익화를 검토 중. 서비스 자체의 영리보다 데이터 확보를 통한 넥슨 전체 사업의 의사결정 품질 향상이 핵심 가치.

\vspace{0.3cm}
\noindent\textbf{문서 구성:} 본 문서는 \textbf{Part~I: 사업 전략} (사업 배경, 데이터 전략, 수익 모델, 리스크)과 \textbf{Part~II: 기술 명세} (시스템 아키텍처, 데이터 스키마, API)로 구성되어 있다.

% =============================================================================
% 목차
% =============================================================================
\tableofcontents
\listoffigures
\listoftables

% =============================================================================
% Part I: 사업 전략
% =============================================================================
\part{사업 전략}

% =============================================================================
% Chapter 1: 사업 배경 및 문제 정의
% =============================================================================
\chapter{사업 배경 및 문제 정의}

\section{스트리밍과 게임 마케팅의 관계}

스트리밍 사업군은 게임에 직접적인 효과가 있는가? 이 질문에 대한 답은 명확히 ``예''이다. 라이브 스트리밍은 게임 마케팅의 핵심 채널로 자리잡았으며, 시청자의 게임 구매 결정, 커뮤니티 형성, 브랜드 인지도에 직접적 영향을 미친다.

그러나 문제는 \textbf{이 효과를 측정할 수 없다}는 점이다. 현재 넥슨이 스트리밍 마케팅에 투자하는 비용 대비 실제 효과를 정량적으로 분석할 수 있는 데이터 인프라가 존재하지 않는다.

\begin{figure}[H]
\centering
\begin{tikzpicture}[node distance=1.5cm]
  \node[box, fill=red!10, draw=red!50, minimum width=3.5cm] (q1) {스트리밍 마케팅\\비용 투입};
  \node[box, fill=gray!20, draw=gray!50, minimum width=3.5cm, right=2cm of q1] (gap) {\textbf{데이터 블랙박스}\\측정 불가 영역};
  \node[box, fill=green!10, draw=green!50, minimum width=3.5cm, right=2cm of gap] (q2) {게임 유입·매출\\효과 (추정)};

  \draw[arrow] (q1) -- (gap);
  \draw[arrow, dashed] (gap) -- (q2);

  \node[below=0.5cm of gap, text=warnred, font=\small\bfseries] {``감각과 경험에 의존''};
\end{tikzpicture}
\caption{현재 넥슨의 스트리밍 마케팅 의사결정 흐름}
\label{fig:current-flow}
\end{figure}

\section{현재 직면한 문제}

스트리밍 사업을 분석하고, 넥슨 게임이 인터넷 방송 신(scene)에서 잘 굴러가게 하려면, 우리는 먼저 ``어떤 데이터가 필요한가''를 정의해야 한다. 그리고 현재 우리가 그 데이터를 갖고 있지 못한 이유를 직시해야 한다.

\begin{table}[H]
\centering
\caption{넥슨이 직면한 3가지 구조적 문제}
\label{tab:problems}
\begin{tabularx}{\textwidth}{L{3cm}X}
\toprule
\textbf{문제} & \textbf{상세 설명} \\
\midrule
직관에 의존한 민심 파악 & 숙제 방송이 아닌 곳에서 어떤 게임이 인기를 얻고 있는지, 유저들이 우리 게임에 대해 어떻게 말하는지 파편화된 정보에 의존해 추측. \\
\addlinespace
데이터 주권의 부재 & 모든 데이터 주권은 각 플랫폼에 종속. 그들이 제공하는 제한적인 데이터만 수동으로 확인 가능. \\
\addlinespace
측정 불가한 ROI & 데이터 부재는 곧 성과 측정 부재. 스트리밍 마케팅 비용이 실제로 얼마나 효과가 있었는지 답할 수 없음. \\
\bottomrule
\end{tabularx}
\end{table}

\section{플랫폼이 데이터를 주지 않는 이유}

스트리밍 데이터를 가져오기 위해 플랫폼에 물어보기 시작했으나, 플랫폼 입장에서는 다음과 같은 이유로 데이터 제공이 어렵다.

\begin{itemize}
  \item \textbf{법률상 문제}: 개인정보보호법에 의해 시청자 행동 데이터의 제3자 제공에 제약.
  \item \textbf{사업전략상 문제}: 타 게임, 타 카테고리 데이터는 플랫폼의 핵심 자산. 경쟁사에 유출될 위험.
  \item \textbf{후원금 데이터}: 플랫폼의 수익 모델 핵심. 외부 공개 시 사업 모델 노출.
\end{itemize}

\begin{table}[H]
\centering
\caption{데이터 수집 시도와 한계}
\label{tab:attempts}
\begin{tabularx}{\textwidth}{L{3cm}L{3cm}X}
\toprule
\textbf{시도} & \textbf{결과} & \textbf{한계} \\
\midrule
공식 API 연동 & 제한적 성공 & 실시간 불가, 카테고리 데이터 없음 \\
제휴/협업 논의 & 진행 지연 & 플랫폼 이해관계 충돌 \\
크롤링/간접 수집 & 법적 리스크 & 정책 위반 우려, 불안정 \\
\bottomrule
\end{tabularx}
\end{table}

\noindent 결론: \textbf{모든 방법은 결국 플랫폼 허락 없이는 불가능하다.} 이는 구조적 한계이며, 접근 방식 자체를 전환해야 한다.


% =============================================================================
% Chapter 2: 분석에 필요한 데이터 정의
% =============================================================================
\chapter{분석에 필요한 데이터 정의}

\section{데이터 수요 분석}

넥슨이 스트리밍 생태계에서 답해야 할 핵심 질문과, 그에 필요한 데이터를 매핑한다.

\begin{table}[H]
\centering
\caption{사업 질문과 필요 데이터 매핑}
\label{tab:data-needs}
\begin{tabularx}{\textwidth}{L{5cm}L{4cm}X}
\toprule
\textbf{사업 질문} & \textbf{필요 데이터} & \textbf{수집 테이블} \\
\midrule
어떤 스트리머가 우리 게임에 진정한 영향력이 있는가? & 시청자 참여율, 후원 금액, 채팅 밀도 & \texttt{viewer\_engagement}, \texttt{events}, \texttt{broadcasts} \\
\addlinespace
경쟁 게임 대비 우리 게임의 방송 비중은? & 카테고리별 방송 수, 시청자 수 & \texttt{platform\_categories}, \texttt{category\_stats} \\
\addlinespace
캠페인 진행 시 실제 시청자 반응은? & 실시간 채팅 감성, 후원 급증 패턴 & \texttt{events}, \texttt{chat\_stats}, \texttt{viewer\_snapshots} \\
\addlinespace
어떤 스트리머를 새로 발굴해야 하는가? & 성장률, 시청자 충성도, 카테고리 다양성 & \texttt{persons}, \texttt{viewer\_engagement}, \texttt{broadcast\_segments} \\
\addlinespace
타 게임 스트리머를 넥슨으로 유입할 수 있는가? & 타 카테고리 방송 데이터, 시청자 이동 패턴 & \texttt{unified\_games}, \texttt{category\_game\_mappings} \\
\bottomrule
\end{tabularx}
\end{table}

\section{수집 가능한 데이터 카테고리}

오버레이를 통해 수집 가능한 데이터는 5개 카테고리로 분류된다.

\subsection{시청자 상호작용 데이터}
\begin{itemize}
  \item \textbf{채팅}: 시청자 닉네임, 메시지 내용, 채팅 시간, 역할(구독자/VIP 등)
  \item \textbf{후원}: 후원자 닉네임, 후원 금액(KRW 정규화), 후원 메시지, 후원 종류(별풍선/치즈 등)
  \item \textbf{구독/팔로우}: 구독·팔로우 시청자, 구독 개월 수, 등급
\end{itemize}

\subsection{방송 상태 데이터}
\begin{itemize}
  \item 방송 시작/종료 시점 및 지속 시간
  \item 크로스 플랫폼 실시간 시청자 수 (분단위 스냅샷)
  \item 방송 카테고리 및 카테고리 변경 이력 (세그먼트)
  \item 방송 제목 변경 추적
\end{itemize}

\subsection{시청자--스트리머 관계 데이터}
\begin{itemize}
  \item 시청자별 채팅 횟수, 후원 횟수, 후원 총액 누적
  \item 시청자 첫 방문/최근 방문 추적
  \item 카테고리별 참여도 분석
\end{itemize}

\subsection{게임/카테고리 트렌드 데이터}
\begin{itemize}
  \item 플랫폼별 카테고리 시청자·스트리머 수 시계열
  \item 크로스 플랫폼 통합 게임 카탈로그
  \item 카테고리 간 시청자 이동 패턴
\end{itemize}

\subsection{세션 데이터}
\begin{itemize}
  \item 시청자 입퇴장 시점 및 체류 시간
  \item 분단위 채팅 통계 (채팅률 변화 추적)
\end{itemize}


% =============================================================================
% Chapter 3: 데이터 수집 전략 — 왜 오버레이인가
% =============================================================================
\chapter{데이터 수집 전략 --- 왜 오버레이인가}

\section{오버레이의 전략적 위치}

이를 꼭 플랫폼에서만 가져와야 하는가? 스트리머들이 쓰는 도구들에 주목했다.

오버레이 시스템은 인터넷 방송에서 필수적으로 쓰는 기능이다. 채팅을 화면에 올리고, 후원의 영상과 이미지를 보이고, 효과음을 낸다. 모든 스트리머는 이미 오버레이를 사용하고 있으며, 이 오버레이를 통해 \textbf{모든 이벤트가 지나간다}.

\begin{figure}[H]
\centering
\begin{tikzpicture}[node distance=1.2cm]
  % 오버레이 (중앙)
  \node[process, minimum width=5cm, minimum height=2cm, fill=accentgreen!20, draw=accentgreen!70, line width=1.5pt] (overlay) {\textbf{\large Streaming Agent}\\(오버레이 --- 브라우저 소스)};

  % 방송 소프트웨어
  \node[adapter, below=2cm of overlay, minimum width=3cm] (obs) {OBS Studio};
  \node[adapter, left=0.5cm of obs, minimum width=3cm] (xsplit) {XSplit};
  \node[adapter, right=0.5cm of obs, minimum width=3cm] (prism) {프릭샷};

  % 플랫폼
  \node[platform, below=2cm of $(xsplit)!0.5!(obs)$, minimum width=2.2cm] (soop) {SOOP};
  \node[platform, below=2cm of obs, minimum width=2.2cm] (chzzk) {치지직};
  \node[platform, below=2cm of $(obs)!0.5!(prism)$, minimum width=2.2cm] (yt) {YouTube};
  \node[platform, right=0.5cm of yt, minimum width=2.2cm] (tw) {Twitch};

  % 시청자
  \node[client, below=2cm of $(chzzk)!0.5!(yt)$, minimum width=6cm, minimum height=1.2cm] (viewer) {\textbf{시청자}};

  % 데이터 수집 (오버레이 오른쪽)
  \node[storage, right=2.5cm of overlay, minimum width=3.5cm] (db) {데이터 수집\\(35 테이블)};
  \node[box, fill=nexonblue!10, draw=nexonblue, below=1cm of db, minimum width=3.5cm] (dash) {넥슨 분석\\대시보드};

  % 오버레이 → 방송 소프트웨어
  \draw[arrow] (overlay) -- (xsplit);
  \draw[arrow] (overlay) -- (obs);
  \draw[arrow] (overlay) -- (prism);

  % 방송 소프트웨어 → 플랫폼
  \draw[arrow] (xsplit) -- (soop);
  \draw[arrow] (obs) -- (chzzk);
  \draw[arrow] (obs) -- (soop);
  \draw[arrow] (obs) -- (yt);
  \draw[arrow] (prism) -- (yt);
  \draw[arrow] (prism) -- (tw);

  % 플랫폼 → 시청자
  \draw[arrow] (soop) -- (viewer);
  \draw[arrow] (chzzk) -- (viewer);
  \draw[arrow] (yt) -- (viewer);
  \draw[arrow] (tw) -- (viewer);

  % 오버레이 → 데이터 수집
  \draw[arrow, dashed, color=nexonblue] (overlay) -- node[above, font=\scriptsize, color=nexonblue] {이벤트 수집} (db);
  \draw[arrow] (db) -- (dash);

  % 레이블
  \node[above=0.3cm of overlay, font=\small\bfseries, color=accentgreen!70!black] {모든 이벤트가 통과하는 중앙 지점};
  \node[left=0.3cm of xsplit, font=\scriptsize, color=gray, align=right] {방송\\소프트웨어};
  \node[left=0.3cm of soop, font=\scriptsize, color=gray, align=right] {스트리밍\\플랫폼};
\end{tikzpicture}
\caption{오버레이의 전략적 위치 --- 방송 송출 플로우에서의 데이터 수집 지점}
\label{fig:data-flow}
\end{figure}

\noindent 오버레이가 데이터 수집에 가장 이상적인 지점인 이유:

\begin{table}[H]
\centering
\begin{tabularx}{\textwidth}{L{3.5cm}X}
\toprule
\textbf{특성} & \textbf{의미} \\
\midrule
스트리머 워크플로우의 중앙 & 방송 시작 전 반드시 설정하는 필수 도구 \\
플랫폼 종속 없음 & 스트리머가 직접 설치, 플랫폼 허락 불필요 \\
항상 켜져 있음 & 방송 전체 시간 동안 실행 \\
모든 이벤트가 통과 & 채팅, 후원, 구독, 팔로우 전부 처리 \\
\bottomrule
\end{tabularx}
\end{table}

\begin{quote}
\textit{``데이터가 플랫폼에 있는 게 아니라, 사실은 스트리머의 화면 위를 다 지나가고 있었다.''}
\end{quote}

\section{플랫폼 독립적 데이터 파이프라인}

Streaming Agent의 핵심 기술은 \textbf{이벤트 정규화(Event Normalization)}이다. 각 플랫폼은 서로 다른 이벤트 형식과 통화 체계를 사용하지만, 이를 하나의 통합 스키마(UnifiedEvent)로 변환한다.

\begin{table}[H]
\centering
\caption{플랫폼별 후원 정규화 매핑}
\label{tab:donation-mapping}
\begin{tabularx}{\textwidth}{L{2cm}L{3cm}L{3cm}X}
\toprule
\textbf{플랫폼} & \textbf{원본 타입} & \textbf{환산 방식} & \textbf{정규화 결과} \\
\midrule
SOOP & 별풍선 & 1개 = 100 KRW & \texttt{amount} (KRW) \\
SOOP & 애드벌룬 & 직접 KRW & \texttt{amount} (KRW) \\
SOOP & 영상풍선 & 직접 KRW & \texttt{amount} (KRW) \\
치지직 & 치즈 & 직접 KRW & \texttt{amount} (KRW) \\
YouTube & Super Chat & USD $\rightarrow$ KRW & \texttt{amount} (KRW) \\
Twitch & Bits & 1 bit = 14 KRW & \texttt{amount} (KRW) \\
\bottomrule
\end{tabularx}
\end{table}

\section{경쟁사 분석 --- 위플랩}

위플랩(Weflab)은 국내 스트리머 오버레이 시장의 사실상 표준이다.

\begin{table}[H]
\centering
\caption{위플랩 vs Streaming Agent 비교}
\label{tab:competitor}
\begin{tabularx}{\textwidth}{L{3.5cm}C{3cm}C{3cm}X}
\toprule
\textbf{항목} & \textbf{위플랩} & \textbf{Streaming Agent} & \textbf{비고} \\
\midrule
직원 수 & 5\textasciitilde15명 & 넥슨 지원 & 지속 투자 가능 \\
오버레이 기능 & 완성도 높음 & 100\% 대체 목표 & 10종 구현 완료 \\
데이터 소유 구조 & 없음 & \textcolor{okgreen}{\textbf{35개 테이블}} & 핵심 차별점 \\
분석 대시보드 & 없음 & \textcolor{okgreen}{\textbf{구현 완료}} & 스트리머 가치 \\
사업 연계 & 불가 & \textcolor{okgreen}{\textbf{넥슨 연동}} & 전략적 가치 \\
BM 안정성 & 취약 & \textcolor{okgreen}{\textbf{넥슨 백업}} & 대형 스트리머 신뢰 \\
\bottomrule
\end{tabularx}
\end{table}

\noindent 위플랩은 기능적 완성도가 높지만, BM이 확고하지 못해 소규모 인원이 운영하는 서비스에 대형 스트리머들마저 기대고 있는 상황이다. 이것은 기회이다.

\begin{quote}
\textit{현재 시장에는 ``게임사 관점에서 스트리밍 데이터를 수집·정규화·활용하기 위한 서비스''가 존재하지 않는다.}
\end{quote}


% =============================================================================
% Chapter 4: 서비스 구축 전략
% =============================================================================
\chapter{서비스 구축 전략}

\section{Phase 1 --- 스트리머 확보}

\subsection{전략 개요}

위플랩보다 더 좋거나 최소 동일한 툴을 제공하여 시장을 장악하고, 스트리밍 데이터를 독점 확보하는 것이 1차 목표이다.

\textbf{절대 원칙:}
\begin{enumerate}
  \item 넥슨 편향적으로 생각하지 않는다. 목표는 \textbf{많은 스트리머가 사용하게 하는 것}이 1순위.
  \item 넥슨 가입 유도나 넥슨이라는 이름이 써있으면, 넥슨 게임 송출 스트리머 외에는 절대 안 쓴다.
  \item 모든 게임사를 연결해주는 기능도 넣어야 한다.
  \item 주기적인 업데이트가 필요하다.
\end{enumerate}

\subsection{스트리머 전환 전략}

스트리머는 설정 바꾸는 것을 귀찮아 한다. 현재 기준 우리 것을 쓰게 하는 방법은 두 가지이다:

\begin{enumerate}
  \item \textbf{쉽게 갈아타게 하기}: 위플랩 설정을 불러오는 마이그레이션 도구 제공
  \item \textbf{우리만의 특별한 기능}: 이걸 써야만 이득을 보게 만들기
\end{enumerate}

\subsection{핵심 전략 4가지}

\subsubsection{넥슨 IP를 활용한 TTS/이펙트}
\begin{itemize}
  \item 넥슨 게임 캐릭터 TTS 보이스 (메이플스토리 세렌, 던전앤파이터 세리아 등)
  \item 다이내믹 모션 이펙트: 정적 GIF를 넘어 방송 화면 전체와 상호작용하는 모션 그래픽
  \item 시청자 이모티콘이 화면에 비처럼 내리거나 터지는 새로운 형태의 참여
\end{itemize}

\subsubsection{스트리머용 분석 대시보드}
어떤 게임, 어떤 장면에서 시청자 반응이 폭발했는지, 어떤 콘텐츠가 구독 전환율이 높았는지 등을 분석하여 데이터 기반 성장 전략 지원.

\subsubsection{설정 템플릿 마켓플레이스}
스트리머뿐 아니라 팬들도 템플릿을 생성할 수 있는 공간. 무료 디자인 공유 및 다운로드.

\subsubsection{스트리머 파트너십 활용}
\begin{itemize}
  \item 넥슨 파트너 스트리머 / 광고 방송 송출 시 Streaming Agent 사용 조건 부여
  \item 신입/소규모 스트리머 발굴 및 지원 프로그램
\end{itemize}

\subsection{구현 완료 오버레이 (10종)}

\begin{table}[H]
\centering
\caption{Streaming Agent 오버레이 목록}
\label{tab:overlays}
\begin{tabularx}{\textwidth}{C{0.8cm}L{2.5cm}X}
\toprule
\textbf{\#} & \textbf{오버레이} & \textbf{주요 기능} \\
\midrule
1 & 채팅창 (Chat) & 26+ 테마, 애니메이션, 역할별 색상, 필터링, 내장 위젯(공지/타이머/시청자수) \\
2 & 후원 알림 (Alert) & TTS, 시그니처 알림, 금액별 커스텀, 룰렛 연동 \\
3 & 후원 자막 (Subtitle) & MVP, 랭킹, 최근 목록, 이미지 후원 표시 \\
4 & 목표치 (Goal) & 바/원형/하트/별/반원 그래프, 자동 증가, 완료 이펙트 \\
5 & 전광판 (Ticker) & 스크롤 메시지, 권한 제한, 명령어 지원 \\
6 & 룰렛 (Roulette) & 세그먼트 커스텀, 트리거 조건, 스핀 애니메이션 \\
7 & 이모지 (Emoji) & Float/Explode/Rain/Bounce 애니메이션 \\
8 & 투표 (Voting) & 최대 10개 옵션, 시간 제한, 실시간 결과 \\
9 & 엔딩 크레딧 (Credits) & 후원자/채팅 참여자 자동 수집, 배경음악 \\
10 & 광고 (Ad) & Banner/Popup/Corner 슬롯, 실시간 성과 측정 \\
\bottomrule
\end{tabularx}
\end{table}

\section{Phase 2 --- 넥슨 독점 기능}

\begin{quote}
\textbf{\color{warnred}주의:} Phase 2는 현재 실행 대상이 아니다. 스트리머 시장의 \textbf{최소 80\% 이상}이 Streaming Agent로 전환된 시점에서 비로소 검토를 시작하는 장기 비전이다. Phase 1의 중립적 포지셔닝이 완전히 확립되기 전에 넥슨 독점 기능을 도입하면 오히려 사용자 이탈을 초래할 수 있다.
\end{quote}

\noindent 충분한 시장 지배력이 확보된 이후, 넥슨 게임의 유저 유입 및 전환을 직접 유도하는 기능을 검토한다:

\begin{itemize}
  \item \textbf{게임--방송 실시간 연동}: 게임 내 이벤트가 방송 오버레이에 실시간 반영
  \item \textbf{스트림 드롭스}: 방송 시청 시 게임 내 보상 제공 (시청자 유입 극대화)
  \item \textbf{인터랙티브 광고}: 오버레이 내 광고 슬롯을 통한 새로운 광고 인벤토리
\end{itemize}

\noindent 이 기능들은 스트리머들이 이미 Streaming Agent에 의존하고 있는 상태에서 ``추가 혜택''으로 제공되어야 의미가 있다. 시기상조에 도입하면 ``넥슨 홍보 도구''라는 인식이 형성되어 서비스 전체의 중립성을 훼손한다.

\section{중립성 유지 전략}

\begin{table}[H]
\centering
\caption{중립성 유지를 위한 핵심 원칙}
\begin{tabularx}{\textwidth}{L{3cm}X}
\toprule
\textbf{원칙} & \textbf{실행 방안} \\
\midrule
넥슨 비노출 & Phase 1에서는 넥슨 브랜딩을 서비스에 노출하지 않음 \\
전 게임사 지원 & 넥슨뿐 아니라 모든 게임사의 카테고리·방송 데이터 수집 \\
중립적 마켓플레이스 & 게임사 무관하게 모든 디자인 템플릿 허용 \\
점진적 독점화 & 충분한 사용자 확보 후 넥슨 IP 기능을 ``추가 혜택''으로 제공 \\
\bottomrule
\end{tabularx}
\end{table}


% =============================================================================
% Chapter 5: 기대효과 및 활용 시나리오 (기존 Ch8에서 이동)
% =============================================================================
\chapter{기대효과 및 활용 시나리오}

\section{데이터의 사업적 가치}

\subsection{마케팅 및 사업개발}

\begin{table}[H]
\centering
\caption{마케팅 관점의 데이터 활용}
\begin{tabularx}{\textwidth}{L{3cm}X}
\toprule
\textbf{활용} & \textbf{설명} \\
\midrule
인플루언서 발굴 & 표면적인 팔로워 수가 아닌, 각 게임에 대한 채팅 참여율과 긍정 반응률을 기반으로 잠재력 있는 스트리머를 경쟁사보다 먼저 발굴 \\
\addlinespace
캠페인 성과 측정 & 특정 스트리머의 방송 이후 채팅에서 우리 게임 언급량 변화, 긍정/부정 키워드를 정량적으로 분석 \\
\addlinespace
정밀 타겟 그룹 식별 & 특정 스트리머의 시청자들이 어떤 게임 성향인지 분석하여 타겟팅 \\
\bottomrule
\end{tabularx}
\end{table}

\subsection{게임 개발 및 운영}

\begin{table}[H]
\centering
\caption{게임 개발/운영 관점의 데이터 활용}
\begin{tabularx}{\textwidth}{L{3cm}X}
\toprule
\textbf{활용} & \textbf{설명} \\
\midrule
유저 피드백 & 실시간 채팅에서 나오는 필터링되지 않은 반응을 즉각 수집 \\
\addlinespace
콘텐츠 매력도 & 스트리머와 시청자가 어떤 보스, 어떤 맵에서 가장 오래 머무는지 분석 \\
\addlinespace
경쟁 게임 분석 & 타 게임 카테고리의 시청자·스트리머 트렌드를 실시간 모니터링 \\
\bottomrule
\end{tabularx}
\end{table}

\section{구체적 활용 시나리오}

\subsection{시나리오 1: 진짜 인플루언서 발굴}

\textbf{데이터 소스}: \texttt{viewer\_engagement} + \texttt{events} + \texttt{persons}

팔로워 수가 아닌 실질적 참여도 기반으로 스트리머를 평가한다:
\begin{lstlisting}[style=sqlstyle]
SELECT p.nickname, p.platform,
       SUM(ve.chat_count) as total_chats,
       SUM(ve.donation_count) as total_donations,
       SUM(ve.total_donation_amount) as total_amount,
       COUNT(DISTINCT ve.person_id) as unique_viewers
FROM persons p
JOIN viewer_engagement ve ON ve.broadcaster_person_id = p.id
WHERE ve.last_seen_at > datetime('now', '-30 days')
GROUP BY p.id
ORDER BY unique_viewers DESC, total_chats DESC;
\end{lstlisting}

\noindent \textbf{비즈니스 인사이트:} 이 쿼리를 통해 ``팔로워는 적지만 실제 시청자 참여도가 높은 스트리머''를 발견할 수 있다. 기존 마케팅에서는 팔로워 수 중심으로 파트너십을 결정했으나, 이 데이터를 통해 \textbf{실질적 영향력} 기반의 의사결정이 가능하다.

\subsection{시나리오 2: 캠페인 ROI 측정}

\textbf{데이터 소스}: \texttt{broadcast\_segments} + \texttt{events} + \texttt{viewer\_snapshots}

특정 게임 캠페인 기간 동안의 시청자 반응 변화를 측정한다:
\begin{lstlisting}[style=sqlstyle]
SELECT bs.category_name,
       COUNT(CASE WHEN e.event_type = 'chat' THEN 1 END) as chat_count,
       COUNT(CASE WHEN e.event_type = 'donation' THEN 1 END) as donation_count,
       AVG(vs.viewer_count) as avg_viewers
FROM broadcast_segments bs
JOIN events e ON e.broadcast_id = bs.broadcast_id
  AND e.event_timestamp BETWEEN bs.segment_started_at AND bs.segment_ended_at
LEFT JOIN viewer_snapshots vs ON vs.broadcast_id = bs.broadcast_id
WHERE bs.category_name LIKE '%MapleStory%'
  AND bs.segment_started_at > '2026-01-01'
GROUP BY bs.category_name;
\end{lstlisting}

\noindent \textbf{비즈니스 인사이트:} 캠페인 집행 전후 채팅량·후원량·시청자 수의 변화를 정량적으로 비교할 수 있다. ``A 스트리머에게 1천만 원을 투자했을 때 시청자 참여가 얼마나 증가했는가''에 대한 구체적 답변이 가능해진다.

\subsection{시나리오 3: 실시간 게임 트렌드 분석}

\textbf{데이터 소스}: \texttt{platform\_categories} + \texttt{unified\_games} + \texttt{category\_stats}

플랫폼 간 게임 인기도를 통합 비교한다. 예를 들어 ``메이플스토리가 SOOP에서는 상위 5위이지만 치지직에서는 10위권 밖''이라는 인사이트를 실시간으로 제공한다.

\subsection{시나리오 4: 시청자 세그먼트 분석}

\textbf{데이터 소스}: \texttt{user\_sessions} + \texttt{viewer\_engagement}

시청자의 체류 시간, 참여 패턴, 이동 경로를 분석하여 타겟 마케팅에 활용한다. ``넥슨 게임 방송을 자주 시청하는 시청자가 주로 보는 다른 게임은 무엇인가''를 파악할 수 있다.


% =============================================================================
% Chapter 6: 수익 모델 및 사업성 분석 (신규)
% =============================================================================
\chapter{사업 모델 분석}

\section{서비스 포지셔닝}

Streaming Agent는 \textbf{영리 목적의 상용 서비스가 아니다.} 핵심 목표는 스트리밍 생태계의 데이터를 넥슨의 전략 자산으로 확보하는 것이며, 서비스 자체는 스트리머에게 최대한 무료로 제공하여 사용자 기반을 빠르게 확보하는 데 집중한다.

\begin{table}[H]
\centering
\caption{서비스 운영 모델}
\label{tab:service-model}
\begin{tabularx}{\textwidth}{L{3.5cm}L{2cm}X}
\toprule
\textbf{구분} & \textbf{과금 여부} & \textbf{설명} \\
\midrule
오버레이 기능 (10종) & 무료 & 채팅, 알림, 룰렛, 투표 등 모든 오버레이 기능을 무료 제공. 위플랩 대비 기능적 우위 확보가 목적 \\
\addlinespace
넥슨 IP 이펙트/TTS & 무료 & 넥슨 캐릭터 TTS, 다이내믹 모션 이펙트 등 넥슨 IP 활용 콘텐츠를 무료 독점 제공하여 차별화 \\
\addlinespace
마켓플레이스 & 무료 & 디자인 템플릿 공유/다운로드를 무료로 운영. 크리에이터와 스트리머 간 커뮤니티 활성화 \\
\addlinespace
분석 대시보드 & 무료 & 스트리머에게 시청자 분석, 콘텐츠 성과 데이터를 무료 제공하여 서비스 충성도 확보 \\
\bottomrule
\end{tabularx}
\end{table}

\noindent 모든 기능을 무료로 제공함으로써 위플랩에서의 이탈 장벽을 최소화하고, 스트리머가 자발적으로 Streaming Agent를 선택하도록 유도한다. 넥슨의 IP와 기술력이 투입된 프리미엄 콘텐츠조차 무료로 제공한다는 점이 시장에서의 결정적 차별화 요소다.

\section{수익 모델: 광고 인벤토리}

현재 검토 중인 유일한 수익 모델은 \textbf{오버레이 내 광고 인벤토리}다.

\subsection{광고의 구조적 차별점}

오버레이에 표시되는 광고는 기존 웹 배너 광고와 근본적으로 다르다:

\begin{itemize}
  \item \textbf{방송 화면 위 직접 노출}: 시청자가 반드시 보게 되는 위치에 광고가 송출됨
  \item \textbf{실시간 컨텍스트}: 방송 카테고리, 시청자 반응, 게임 종류에 따라 타겟팅 가능
  \item \textbf{높은 주목도}: 시청자가 능동적으로 시청 중인 콘텐츠 위에 노출되므로 기존 디스플레이 광고 대비 주목도가 높음
  \item \textbf{스트리머 수익 분배}: 스트리머와 수익을 분배하여 자발적 참여 유도 (예: 70:30 비율)
\end{itemize}

\subsection{광고 슬롯 유형}

\begin{table}[H]
\centering
\caption{광고 슬롯 유형}
\begin{tabularx}{\textwidth}{L{2.5cm}X}
\toprule
\textbf{유형} & \textbf{설명} \\
\midrule
Banner & 방송 화면 상/하단에 배너 형태로 노출 \\
\addlinespace
Popup & 특정 이벤트(후원, 구독 등) 발생 시 팝업 형태로 노출 \\
\addlinespace
Corner & 화면 모서리에 소형 광고 상시 노출 \\
\bottomrule
\end{tabularx}
\end{table}

\subsection{광고 과금 모델}

CPM(1,000회 노출당 과금) 또는 CPC(클릭당 과금) 방식을 검토 중이다. 오버레이 특성상 노출이 보장되므로 CPM 모델이 적합할 가능성이 높다.

\section{시장 환경}

\begin{table}[H]
\centering
\caption{국내 스트리밍 오버레이 시장 현황}
\label{tab:market-size}
\begin{tabularx}{\textwidth}{L{3cm}X}
\toprule
\textbf{항목} & \textbf{현황} \\
\midrule
주요 플랫폼 & SOOP(구 아프리카TV), 네이버 치지직 \\
\addlinespace
활성 스트리머 규모 & 수만 명 (양 플랫폼 합산 추정) \\
\addlinespace
기존 오버레이 서비스 & 위플랩 (사실상 독점, 직원 5\textasciitilde15명, 추정 매출 3\textasciitilde10억 원) \\
\addlinespace
경쟁 구도 & 경쟁자 부재로 시장 성장 정체. 게임사 관점의 오버레이 서비스 전무 \\
\addlinespace
진입 기회 & 넥슨의 IP·자본력·파트너십을 활용하면 빠른 시장 점유 가능 \\
\bottomrule
\end{tabularx}
\end{table}

\noindent 위플랩이 독점하고 있는 시장에 넥슨이 무료 + 고품질로 진입하면, 스트리머 입장에서 전환하지 않을 이유가 없다. 데이터 확보를 위한 사용자 기반 구축이 최우선이며, 광고 수익화는 충분한 사용자가 확보된 이후에 본격적으로 추진한다.

\section{비용 구조}

\begin{table}[H]
\centering
\caption{연간 비용 구조 추정}
\label{tab:cost-structure}
\begin{tabularx}{\textwidth}{L{3cm}C{2.5cm}X}
\toprule
\textbf{항목} & \textbf{연간 추정} & \textbf{상세} \\
\midrule
개발 인력 & 4\textasciitilde6억 원 & 프론트엔드 2명, 백엔드 2명, 데이터 1명, PM 1명 \\
\addlinespace
인프라 & 1\textasciitilde2억 원 & 클라우드 서버, DB, CDN, 스토리지, 모니터링 \\
\addlinespace
운영 & 0.5\textasciitilde1억 원 & 고객 지원, QA, 커뮤니티 관리 \\
\addlinespace
마케팅 & 1\textasciitilde2억 원 & 스트리머 파트너십, 프로모션, 이벤트 \\
\midrule
\textbf{합계} & \textbf{6.5\textasciitilde11억 원} & \\
\bottomrule
\end{tabularx}
\end{table}

\noindent 이 비용은 넥슨의 전략적 투자로 집행된다. 서비스 자체의 독립 수익으로 운영비를 충당하는 것이 1차 목표가 아니라, \textbf{데이터 확보를 통한 넥슨 전체 사업의 의사결정 품질 향상}이 본질적 투자 회수 방식이다.


% =============================================================================
% Chapter 7: 리스크 및 대응 전략
% =============================================================================
\chapter{리스크 및 대응 전략}

\section{사업 리스크 분석}

\begin{table}[H]
\centering
\caption{사업 리스크 분석 및 대응 전략}
\label{tab:biz-risks}
\begin{tabularx}{\textwidth}{L{3cm}L{2cm}X}
\toprule
\textbf{리스크} & \textbf{심각도} & \textbf{대응 방안} \\
\midrule
시장 저항 및 중립성 훼손 & 높음 & Phase 1에서는 철저히 중립적 툴로 포지셔닝. 넥슨 관련 기능은 부가적 기능으로 명분 확보 \\
\addlinespace
스트리머 모객 & 높음 & 위플랩 마이그레이션 도구 제공, 마켓플레이스로 트렌디한 템플릿 자동 생성 \\
\addlinespace
플랫폼 정책 변경 & 중간 & 오버레이 자체가 플랫폼 독립적이므로 구조적 리스크 낮음. WebSocket 연결 방식은 모니터링 \\
\addlinespace
데이터 보안/개인정보 & 중간 & 개인정보보호 규정 설계, 익명 데이터 수집, 활용 목적 투명 고지. 최고 수준의 보안 체계 \\
\addlinespace
광고 수익 모델 불확실성 & 중간 & 광고 수익화는 충분한 사용자 확보 후 검토. 서비스 자체 수익보다 데이터 전략 자산 확보가 본질적 투자 가치 \\
\addlinespace
경쟁사 대응 & 낮음 & 위플랩은 소규모 팀으로 넥슨 IP 및 데이터 인프라 대응 불가. 글로벌 서비스는 한국 시장 특성상 진입 장벽 높음 \\
\bottomrule
\end{tabularx}
\end{table}

\section{리스크 완화 전략}

\begin{enumerate}
  \item \textbf{중립성 유지}: 출시 후 최소 1년간은 ``넥슨''이라는 이름을 서비스 전면에 노출하지 않는다. 모든 게임사의 방송을 동등하게 지원하여 시장 신뢰를 확보한다.
  \item \textbf{완전 무료 전략}: 모든 기능을 무료로 제공하여 진입 장벽을 제거한다. 광고 수익화는 사용자 기반이 충분히 확보된 이후 검토한다.
  \item \textbf{데이터 투명성}: 수집하는 데이터의 범위와 활용 목적을 명확히 고지하고, 스트리머가 데이터 수집을 거부할 수 있는 옵션을 제공한다.
  \item \textbf{지속적 혁신}: 마켓플레이스와 커뮤니티를 통해 사용자 피드백을 빠르게 반영하고, 위플랩 대비 기능적 우위를 유지한다.
\end{enumerate}


% =============================================================================
% Chapter 8: 실행 로드맵
% =============================================================================
\chapter{실행 로드맵}

\section{단계별 실행 계획}

\begin{table}[H]
\centering
\caption{단계별 실행 계획 요약}
\label{tab:exec-roadmap}
\begin{tabularx}{\textwidth}{L{2.5cm}L{2.5cm}X}
\toprule
\textbf{단계} & \textbf{목표 시점} & \textbf{핵심 마일스톤} \\
\midrule
현재 (완료) & 2026 Q1 & 10종 오버레이, 35개 DB 테이블, 50+ API, 실시간 파이프라인 구축 \\
\addlinespace
Phase 1 전반 & 2026 Q2\textasciitilde Q3 & 넥슨 IP TTS 출시, 위플랩 마이그레이션 도구, 마켓플레이스 오픈 \\
\addlinespace
Phase 1 후반 & 2026 Q4\textasciitilde 2027 Q1 & 분석 대시보드 고도화, 게임별 성과 리포트, 광고 시스템 검증 \\
\addlinespace
Phase 2 & 2027 Q2\textasciitilde & 넥슨 독점 기능 (스트림 드롭스, 인게임 연동), 광고 인벤토리 본격 운영 \\
\bottomrule
\end{tabularx}
\end{table}

\section{핵심 KPI}

\begin{table}[H]
\centering
\caption{단계별 핵심 KPI}
\begin{tabularx}{\textwidth}{L{3.5cm}C{3cm}C{3cm}}
\toprule
\textbf{KPI} & \textbf{Phase 1 목표} & \textbf{Phase 2 목표} \\
\midrule
위플랩 대비 점유율 & 30\% & 70\% \\
일일 수집 이벤트 & 100만 건 & 500만 건 \\
월 활성 시청자(MAU) & 50만 명 & 200만 명 \\
데이터 활용 부서 수 & 3개 부서 & 10개 부서 \\
광고 시스템 & 검증 완료 & 본격 운영 \\
\bottomrule
\end{tabularx}
\end{table}

\section{투자 대비 기대 효과}

넥슨이 이 프로젝트에 투자해야 하는 이유를 정리한다:

\begin{enumerate}
  \item \textbf{데이터 독점}: 경쟁사가 접근할 수 없는 실시간 스트리밍 데이터를 넥슨만의 전략 자산으로 확보
  \item \textbf{마케팅 ROI 측정}: 감각이 아닌 데이터 기반의 스트리밍 마케팅 의사결정 가능
  \item \textbf{생태계 지배력}: 스트리밍 인프라를 장악하여 넥슨 게임의 방송 노출·유입 극대화
  \item \textbf{방어적 가치}: 경쟁 게임사가 유사 서비스를 구축하기 전에 시장 선점
  \item \textbf{광고 인프라 확보}: 사용자 기반 확보 후 오버레이 광고라는 새로운 광고 채널을 운영할 수 있는 기반 마련
\end{enumerate}


% =============================================================================
% Part II: 기술 명세
% =============================================================================
\part{기술 명세}

% =============================================================================
% Chapter 9: 시스템 아키텍처 (기존 Ch5)
% =============================================================================
\chapter{시스템 아키텍처}

\section{전체 시스템 구조}

\begin{table}[H]
\centering
\caption{기술 스택}
\label{tab:techstack}
\begin{tabularx}{\textwidth}{L{3cm}X}
\toprule
\textbf{영역} & \textbf{기술} \\
\midrule
Frontend & React 19, Vite 7, React Router DOM 7, Lucide React, Recharts \\
Backend & Express 5, Socket.io 4, SQLite3 \\
인증 & JWT (jsonwebtoken), bcrypt, OAuth (SOOP, Naver, Google, Twitch) \\
스타일 & CSS Custom Properties, Glass-morphism \\
로깅 & Pino \\
캐싱 & Redis (선택) \\
\bottomrule
\end{tabularx}
\end{table}

\begin{figure}[H]
\centering
\begin{tikzpicture}[node distance=0.7cm, scale=0.9, transform shape]
  % 클라이언트 레이어
  \node[client, minimum width=12cm, minimum height=1.2cm] (client) {\textbf{Streamer Client} (React 19 + Vite 7)};

  % 화살표 1
  \node[below=0.7cm of client] (arr1) {};
  \draw[arrow] (client.south) -- (arr1.center);

  % 오버레이 레이어 — 타이틀을 박스 상단 내부에 배치
  \node[box, fill=yellow!5, draw=yellow!50, below=0.1cm of arr1, minimum width=12cm, minimum height=2.2cm] (overlay) {};
  \node[font=\small\bfseries] at ($(overlay.north)+(0,-0.35)$) {Overlay / Widget Layer (10종)};

  \node[box, fill=yellow!20, draw=yellow!60, minimum width=1.3cm, minimum height=0.7cm, font=\scriptsize] at ($(overlay.south)+(- 4.5,0.55)$) {Chat};
  \node[box, fill=yellow!20, draw=yellow!60, minimum width=1.3cm, minimum height=0.7cm, font=\scriptsize] at ($(overlay.south)+(-3,0.55)$) {Alert};
  \node[box, fill=yellow!20, draw=yellow!60, minimum width=1.3cm, minimum height=0.7cm, font=\scriptsize] at ($(overlay.south)+(-1.5,0.55)$) {Goal};
  \node[box, fill=yellow!20, draw=yellow!60, minimum width=1.3cm, minimum height=0.7cm, font=\scriptsize] at ($(overlay.south)+(0,0.55)$) {Ticker};
  \node[box, fill=yellow!20, draw=yellow!60, minimum width=1.3cm, minimum height=0.7cm, font=\scriptsize] at ($(overlay.south)+(1.5,0.55)$) {Vote};
  \node[box, fill=yellow!20, draw=yellow!60, minimum width=1.3cm, minimum height=0.7cm, font=\scriptsize] at ($(overlay.south)+(3,0.55)$) {Emoji};
  \node[box, fill=yellow!20, draw=yellow!60, minimum width=1.3cm, minimum height=0.7cm, font=\scriptsize] at ($(overlay.south)+(4.5,0.55)$) {+4};

  % 화살표 2
  \node[below=0.7cm of overlay] (arr2) {};
  \draw[arrow] (overlay.south) -- (arr2.center);

  % 서버 레이어
  \node[process, below=0.1cm of arr2, minimum width=12cm, minimum height=1.2cm] (server) {\textbf{Express 5 + Socket.io 4} (Event Ingestion Layer)};

  % 화살표 3 (양방향)
  \node[below=0.7cm of server] (arr3) {};
  \draw[biarrow] (server.south) -- (arr3.center);

  % 어댑터 — 타이틀을 박스 상단 내부에 배치
  \node[adapter, below=0.1cm of arr3, minimum width=12cm, minimum height=1.8cm] (adapters) {};
  \node[font=\small\bfseries] at ($(adapters.north)+(0,-0.35)$) {Platform Adapters};
  \node[font=\scriptsize] at ($(adapters.south)+(-3.5,0.45)$) {\textcolor{soop}{\textbf{SOOP}} (WebSocket)};
  \node[font=\scriptsize] at ($(adapters.south)+(-0.8,0.45)$) {\textcolor{chzzk}{\textbf{Chzzk}} (WebSocket)};
  \node[font=\scriptsize] at ($(adapters.south)+(2,0.45)$) {\textcolor{youtube}{\textbf{YouTube}} (API)};
  \node[font=\scriptsize] at ($(adapters.south)+(4.5,0.45)$) {\textcolor{twitch}{\textbf{Twitch}} (EventSub)};

  % 화살표 4
  \node[below=0.7cm of adapters] (arr4) {};
  \draw[arrow] (adapters.south) -- (arr4.center);

  % 정규화
  \node[process, below=0.1cm of arr4, minimum width=12cm, fill=green!15] (normalizer) {\textbf{EventNormalizer} --- 플랫폼별 이벤트 $\rightarrow$ UnifiedEvent Schema};

  % 화살표 5
  \node[below=0.7cm of normalizer] (arr5) {};
  \draw[arrow] (normalizer.south) -- (arr5.center);

  % 저장
  \node[storage, below=0.1cm of arr5, minimum width=12cm, minimum height=1.2cm] (storage) {\textbf{SQLite3 (unified.db)} --- 35개 테이블, 50+ 인덱스};
\end{tikzpicture}
\caption{시스템 레이어 아키텍처}
\label{fig:architecture}
\end{figure}

\section{플랫폼 어댑터}

각 플랫폼은 서로 다른 프로토콜과 이벤트 형식을 사용한다. Streaming Agent는 \texttt{BaseAdapter} 패턴을 통해 플랫폼별 차이를 추상화한다.

\begin{table}[H]
\centering
\caption{플랫폼별 어댑터 특성}
\label{tab:adapters}
\begin{tabularx}{\textwidth}{L{2cm}L{2.5cm}L{2.5cm}X}
\toprule
\textbf{플랫폼} & \textbf{프로토콜} & \textbf{이벤트 타입} & \textbf{특이사항} \\
\midrule
SOOP & WebSocket & chat, donation (5종), follow & 별풍선/애드벌룬/영상풍선/미션/스티커 구분 \\
치지직 & WebSocket & chat, donation, subscribe & 치즈 단일 통화 \\
YouTube & REST API & chat, super\_chat, member & 다국가 통화 환산 필요 \\
Twitch & EventSub & chat, bits, subscribe, follow & Bits 환산 (1bit = 14KRW) \\
\bottomrule
\end{tabularx}
\end{table}

\section{이벤트 정규화 파이프라인}

모든 플랫폼 이벤트는 다음의 UnifiedEvent 스키마로 정규화된다:

\begin{lstlisting}[style=jsstyle, caption={UnifiedEvent 스키마}, label={lst:unified-event}]
{
  id: UUID,                    // 고유 이벤트 ID
  type: 'chat' | 'donation' | 'subscribe' | 'follow',
  platform: 'chzzk' | 'soop' | 'youtube' | 'twitch',
  sender: {
    id: string,                // 플랫폼 사용자 ID
    nickname: string,          // 표시 닉네임
    role: 'streamer' | 'manager' | 'vip' | 'fan' | 'regular',
    profileImage: string       // 프로필 이미지 URL
  },
  content: {
    message: string,           // 메시지 내용
    amount: number,            // KRW 환산 금액
    originalAmount: number,    // 원본 금액
    currency: string,          // 원본 통화
    donationType: string       // 플랫폼별 후원 타입
  },
  metadata: {
    timestamp: ISO8601,        // 이벤트 발생 시각
    channelId: string,         // 채널 ID
    broadcastId: string,       // 방송 ID
    categoryId: string,        // 카테고리 ID
    categoryName: string       // 카테고리 이름
  }
}
\end{lstlisting}


% =============================================================================
% Chapter 6: 데이터 스키마 상세
% =============================================================================
\chapter{데이터 스키마 상세}

본 장은 Streaming Agent가 수집·저장하는 모든 데이터의 구조를 상세히 기술한다. 총 35개 테이블이 6개 그룹으로 분류된다.

\section{스키마 개요}

\begin{figure}[H]
\centering
\begin{tikzpicture}[node distance=1.2cm, scale=0.85, transform shape]
  % 상단 행: 코어(좌), 카테고리(우)
  \node[box, fill=blue!10, draw=blue!50, minimum width=6cm, minimum height=3cm] (core) {};
  \node[above=0.05cm] at (core.north) {\small\textbf{코어 스트리밍 (9)}};
  \node[font=\scriptsize, align=left] at (core.center) {%
    persons, events\\broadcasts, broadcast\_segments\\categories, viewer\_engagement\\viewer\_snapshots\\user\_sessions, chat\_stats};

  \node[box, fill=orange!10, draw=orange!50, minimum width=6cm, minimum height=2.2cm, right=1.5cm of core] (cat) {};
  \node[above=0.05cm] at (cat.north) {\small\textbf{카테고리/게임 (5)}};
  \node[font=\scriptsize, align=left] at (cat.center) {%
    platform\_categories\\unified\_games\\category\_game\_mappings\\category\_stats, viewer\_stats};

  % 중단 행: 오버레이(좌), 사용자(우)
  \node[box, fill=green!10, draw=green!50, minimum width=6cm, minimum height=3cm, below=1.5cm of core] (overlay) {};
  \node[above=0.05cm] at (overlay.north) {\small\textbf{오버레이 설정 (10)}};
  \node[font=\scriptsize, align=left] at (overlay.center) {%
    settings, user\_settings\\roulette\_wheels, signature\_sounds\\emoji\_settings\\voting\_polls, poll\_votes\\ending\_credits\\chat\_bots, bot\_commands\\bot\_auto\_messages};

  \node[box, fill=purple!10, draw=purple!50, minimum width=6cm, minimum height=1.8cm, below=1.5cm of cat] (auth) {};
  \node[above=0.05cm] at (auth.north) {\small\textbf{사용자/인증 (4)}};
  \node[font=\scriptsize, align=left] at (auth.center) {%
    users, refresh\_tokens\\token\_blacklist\\user\_sessions};

  % 하단 행: 광고(좌), 마켓플레이스(우)
  \node[box, fill=red!10, draw=red!50, minimum width=6cm, minimum height=1.8cm, below=1.5cm of overlay] (ad) {};
  \node[above=0.05cm] at (ad.north) {\small\textbf{광고 시스템 (4)}};
  \node[font=\scriptsize, align=left] at (ad.center) {%
    ad\_slots, ad\_campaigns\\ad\_impressions\\ad\_settlements};

  \node[box, fill=yellow!10, draw=yellow!50, minimum width=6cm, minimum height=1.5cm, below=1.5cm of auth] (market) {};
  \node[above=0.05cm] at (market.north) {\small\textbf{마켓플레이스 (3)}};
  \node[font=\scriptsize, align=left] at (market.center) {%
    creators, designs\\design\_reviews};

  % 연결선
  \draw[dashed, gray, thick] (core.east) -- (cat.west);
  \draw[dashed, gray, thick] (core.south) -- (overlay.north);
  \draw[dashed, gray, thick] (auth.west) -- (overlay.east);
  \draw[dashed, gray, thick] (auth.south) -- (market.north);
  \draw[dashed, gray, thick] (overlay.south) -- (ad.north);
\end{tikzpicture}
\caption{데이터베이스 스키마 그룹 구조 (35개 테이블)}
\label{fig:schema-overview}
\end{figure}

\begin{table}[H]
\centering
\caption{테이블 그룹 요약}
\label{tab:table-summary}
\begin{tabularx}{\textwidth}{L{3.5cm}C{1.5cm}X}
\toprule
\textbf{그룹} & \textbf{테이블 수} & \textbf{목적} \\
\midrule
코어 스트리밍 & 9 & 이벤트, 방송, 시청자, 세그먼트 등 핵심 데이터 \\
카테고리/게임 & 5 & 크로스 플랫폼 카테고리 통합 및 트렌드 \\
오버레이 설정 & 10 & 10종 오버레이 위젯의 설정 데이터 \\
사용자/인증 & 4 & 계정, JWT 토큰, 세션 관리 \\
광고 시스템 & 4 & 광고 슬롯, 캠페인, 노출/클릭, 정산 \\
마켓플레이스 & 3 & 디자인 크리에이터, 템플릿, 리뷰 \\
\bottomrule
\end{tabularx}
\end{table}

% ---- 6.2 코어 스트리밍 데이터 ----
\section{코어 스트리밍 데이터}

\subsection{persons --- 인물 통합 테이블}

스트리머와 시청자를 하나의 테이블에서 관리한다. \texttt{channel\_id}의 유무로 방송인과 시청자를 구분한다.

\begin{longtable}{L{3.5cm}L{2.5cm}L{2cm}L{5.5cm}}
\caption{persons 테이블 (15 컬럼)} \label{tab:persons} \\
\toprule
\textbf{컬럼} & \textbf{타입} & \textbf{제약} & \textbf{설명} \\
\midrule
\endfirsthead
\toprule
\textbf{컬럼} & \textbf{타입} & \textbf{제약} & \textbf{설명} \\
\midrule
\endhead
id & INTEGER & PK, AUTO & 고유 식별자 \\
platform & TEXT & NOT NULL & soop $|$ chzzk $|$ twitch $|$ youtube \\
platform\_user\_id & TEXT & NOT NULL & 플랫폼 내 사용자 ID \\
nickname & TEXT & & 표시 닉네임 \\
profile\_image\_url & TEXT & & 프로필 이미지 URL \\
channel\_id & TEXT & & 채널 ID (NULL = 시청자) \\
channel\_description & TEXT & & 채널 소개 \\
follower\_count & INTEGER & DEFAULT 0 & 팔로워 수 \\
subscriber\_count & INTEGER & DEFAULT 0 & 구독자 수 \\
total\_broadcast\_minutes & INTEGER & DEFAULT 0 & 총 방송 시간 (분) \\
last\_broadcast\_at & DATETIME & & 마지막 방송 시각 \\
first\_seen\_at & DATETIME & DEFAULT NOW & 최초 발견 시각 \\
last\_seen\_at & DATETIME & DEFAULT NOW & 최근 활동 시각 \\
created\_at & DATETIME & DEFAULT NOW & 생성 시각 \\
updated\_at & DATETIME & DEFAULT NOW & 수정 시각 \\
\bottomrule
\multicolumn{4}{l}{\small UNIQUE(platform, platform\_user\_id)} \\
\end{longtable}

\subsection{events --- 이벤트 통합 테이블}

채팅, 후원, 구독, 팔로우 등 모든 이벤트를 하나의 테이블에 기록한다. UUID 기반 PK로 중복 방지.

\begin{longtable}{L{3.5cm}L{2.5cm}L{2cm}L{5.5cm}}
\caption{events 테이블 (16 컬럼)} \label{tab:events} \\
\toprule
\textbf{컬럼} & \textbf{타입} & \textbf{제약} & \textbf{설명} \\
\midrule
\endfirsthead
\toprule
\textbf{컬럼} & \textbf{타입} & \textbf{제약} & \textbf{설명} \\
\midrule
\endhead
id & TEXT & PK & UUID v4 \\
event\_type & TEXT & NOT NULL & chat $|$ donation $|$ subscribe $|$ follow $|$ view \\
platform & TEXT & NOT NULL & soop $|$ chzzk $|$ twitch $|$ youtube \\
actor\_person\_id & INTEGER & FK$\rightarrow$persons & 이벤트 발생 주체 \\
actor\_nickname & TEXT & & 비정규화된 닉네임 (빠른 조회) \\
actor\_role & TEXT & & streamer $|$ manager $|$ vip $|$ fan $|$ regular \\
target\_person\_id & INTEGER & FK$\rightarrow$persons & 이벤트 대상 (스트리머) \\
target\_channel\_id & TEXT & NOT NULL & 대상 채널 ID \\
broadcast\_id & INTEGER & FK$\rightarrow$broadcasts & 방송 ID \\
message & TEXT & & 채팅/후원 메시지 \\
amount & INTEGER & & KRW 환산 금액 \\
original\_amount & INTEGER & & 원본 금액 \\
currency & TEXT & & 원본 통화 \\
donation\_type & TEXT & & 별풍선/치즈/bits 등 \\
event\_timestamp & DATETIME & NOT NULL & 이벤트 발생 시각 \\
ingested\_at & DATETIME & DEFAULT NOW & 수집 시각 \\
\bottomrule
\end{longtable}

\subsection{broadcasts --- 방송 테이블}

\begin{longtable}{L{3.5cm}L{2.5cm}L{2cm}L{5.5cm}}
\caption{broadcasts 테이블 (18 컬럼)} \label{tab:broadcasts} \\
\toprule
\textbf{컬럼} & \textbf{타입} & \textbf{제약} & \textbf{설명} \\
\midrule
\endfirsthead
\toprule
\textbf{컬럼} & \textbf{타입} & \textbf{제약} & \textbf{설명} \\
\midrule
\endhead
id & INTEGER & PK, AUTO & 고유 식별자 \\
platform & TEXT & NOT NULL & 플랫폼 \\
channel\_id & TEXT & NOT NULL & 채널 ID \\
broadcast\_id & TEXT & NOT NULL & 플랫폼 방송 ID \\
broadcaster\_person\_id & INTEGER & FK$\rightarrow$persons & 방송인 \\
title & TEXT & & 방송 제목 \\
thumbnail\_url & TEXT & & 썸네일 \\
current\_viewer\_count & INTEGER & DEFAULT 0 & 현재 시청자 수 \\
peak\_viewer\_count & INTEGER & DEFAULT 0 & 최대 시청자 수 \\
avg\_viewer\_count & INTEGER & DEFAULT 0 & 평균 시청자 수 \\
viewer\_sum & INTEGER & DEFAULT 0 & 시청자 합산 (평균 계산용) \\
snapshot\_count & INTEGER & DEFAULT 0 & 스냅샷 횟수 \\
is\_live & INTEGER & DEFAULT 1 & 라이브 여부 \\
started\_at & DATETIME & & 방송 시작 \\
ended\_at & DATETIME & & 방송 종료 \\
duration\_minutes & INTEGER & & 방송 시간 (분) \\
recorded\_at & DATETIME & DEFAULT NOW & 기록 시각 \\
updated\_at & DATETIME & DEFAULT NOW & 수정 시각 \\
\bottomrule
\multicolumn{4}{l}{\small UNIQUE(platform, channel\_id, broadcast\_id)} \\
\end{longtable}

\subsection{broadcast\_segments --- 카테고리 변경 세그먼트}

방송 중 카테고리(게임)가 변경될 때마다 새로운 세그먼트가 생성된다. 이를 통해 ``어떤 게임을 얼마나 했는지''를 정확히 추적할 수 있다.

\begin{longtable}{L{3.5cm}L{2.5cm}L{2cm}L{5.5cm}}
\caption{broadcast\_segments 테이블 (10 컬럼)} \\
\toprule
\textbf{컬럼} & \textbf{타입} & \textbf{제약} & \textbf{설명} \\
\midrule
id & INTEGER & PK, AUTO & 고유 식별자 \\
broadcast\_id & INTEGER & FK, CASCADE & 방송 ID \\
platform & TEXT & NOT NULL & 플랫폼 \\
channel\_id & TEXT & NOT NULL & 채널 ID \\
category\_id & TEXT & & 카테고리 ID \\
category\_name & TEXT & & 카테고리 이름 \\
segment\_started\_at & DATETIME & NOT NULL & 세그먼트 시작 \\
segment\_ended\_at & DATETIME & & 세그먼트 종료 \\
peak\_viewer\_count & INTEGER & DEFAULT 0 & 구간 최대 시청자 \\
avg\_viewer\_count & INTEGER & DEFAULT 0 & 구간 평균 시청자 \\
\bottomrule
\end{longtable}

\subsection{viewer\_engagement --- 시청자--방송인 관계}

특정 시청자가 특정 방송인의 채널에서 보인 참여도를 누적 기록한다.

\begin{longtable}{L{3.5cm}L{2.5cm}L{2cm}L{5.5cm}}
\caption{viewer\_engagement 테이블 (12 컬럼)} \\
\toprule
\textbf{컬럼} & \textbf{타입} & \textbf{제약} & \textbf{설명} \\
\midrule
id & INTEGER & PK, AUTO & 고유 식별자 \\
person\_id & INTEGER & FK$\rightarrow$persons & 시청자 \\
platform & TEXT & NOT NULL & 플랫폼 \\
channel\_id & TEXT & NOT NULL & 채널 ID \\
broadcaster\_person\_id & INTEGER & FK$\rightarrow$persons & 방송인 \\
category\_id & TEXT & & 카테고리 \\
chat\_count & INTEGER & DEFAULT 0 & 채팅 횟수 누적 \\
donation\_count & INTEGER & DEFAULT 0 & 후원 횟수 누적 \\
total\_donation\_amount & INTEGER & DEFAULT 0 & 후원 총액 (KRW) \\
first\_seen\_at & DATETIME & DEFAULT NOW & 첫 참여 시각 \\
last\_seen\_at & DATETIME & DEFAULT NOW & 마지막 참여 \\
updated\_at & DATETIME & DEFAULT NOW & 수정 시각 \\
\bottomrule
\multicolumn{4}{l}{\small UNIQUE(person\_id, channel\_id, platform, category\_id)} \\
\end{longtable}

\subsection{viewer\_snapshots --- 시청자 수 시계열}

분단위로 시청자 수와 채팅률을 스냅샷으로 기록한다.

\begin{longtable}{L{3.5cm}L{2.5cm}L{2cm}L{5.5cm}}
\caption{viewer\_snapshots 테이블 (9 컬럼)} \\
\toprule
\textbf{컬럼} & \textbf{타입} & \textbf{제약} & \textbf{설명} \\
\midrule
id & INTEGER & PK, AUTO & 고유 식별자 \\
platform & TEXT & NOT NULL & 플랫폼 \\
channel\_id & TEXT & NOT NULL & 채널 ID \\
broadcast\_id & INTEGER & FK & 방송 ID \\
segment\_id & INTEGER & FK & 세그먼트 ID \\
viewer\_count & INTEGER & NOT NULL & 시청자 수 \\
chat\_rate\_per\_minute & INTEGER & & 분당 채팅 수 \\
snapshot\_at & DATETIME & NOT NULL & 스냅샷 시각 \\
ingested\_at & DATETIME & DEFAULT NOW & 수집 시각 \\
\bottomrule
\end{longtable}

\subsection{chat\_stats --- 분단위 채팅 통계}

유저별 분단위 채팅 횟수를 집계한다.

\begin{longtable}{L{3.5cm}L{2.5cm}L{2cm}L{5.5cm}}
\caption{chat\_stats 테이블 (8 컬럼)} \\
\toprule
\textbf{컬럼} & \textbf{타입} & \textbf{제약} & \textbf{설명} \\
\midrule
id & BIGINT & PK, AUTO & 고유 식별자 \\
platform & VARCHAR(20) & NOT NULL & 플랫폼 \\
channel\_id & VARCHAR(255) & NOT NULL & 채널 ID \\
user\_id & VARCHAR(255) & NOT NULL & 유저 ID \\
user\_nickname & VARCHAR(255) & & 유저 닉네임 \\
minute\_timestamp & TIMESTAMP & NOT NULL & 분단위 타임스탬프 \\
chat\_count & INTEGER & DEFAULT 1 & 해당 분 채팅 수 \\
created\_at & TIMESTAMP & DEFAULT NOW & 생성 시각 \\
\bottomrule
\multicolumn{4}{l}{\small UNIQUE(platform, channel\_id, user\_id, minute\_timestamp)} \\
\end{longtable}

% ---- 6.3 게임/카테고리 카탈로그 ----
\section{게임/카테고리 카탈로그}

크로스 플랫폼 카테고리 통합은 ``타 게임, 타 카테고리 방송 데이터 수집''이라는 핵심 목표를 달성하기 위한 핵심 구조이다.

\begin{figure}[H]
\centering
\begin{tikzpicture}[node distance=1.5cm]
  \node[platform, minimum width=3cm] (soop-cat) {SOOP\\카테고리};
  \node[platform, minimum width=3cm, right=2cm of soop-cat] (chzzk-cat) {치지직\\카테고리};

  \node[process, below=1.5cm of $(soop-cat)!0.5!(chzzk-cat)$, minimum width=4cm] (mapping) {category\_game\\\_mappings\\(confidence score)};

  \node[storage, below=1.5cm of mapping, minimum width=4cm] (unified) {unified\_games\\통합 게임 카탈로그};

  \draw[arrow] (soop-cat) -- (mapping);
  \draw[arrow] (chzzk-cat) -- (mapping);
  \draw[arrow] (mapping) -- (unified);

  \node[right=1cm of mapping, font=\scriptsize, color=gray, align=left] {SOOP ``리그 오브 레전드''\\= 치지직 ``League of Legends''\\$\rightarrow$ unified\_games.id = 42};
\end{tikzpicture}
\caption{크로스 플랫폼 카테고리 통합 흐름}
\label{fig:category-unification}
\end{figure}

\begin{longtable}{L{3.5cm}L{2.5cm}L{2cm}L{5.5cm}}
\caption{platform\_categories 테이블 (11 컬럼)} \\
\toprule
\textbf{컬럼} & \textbf{타입} & \textbf{제약} & \textbf{설명} \\
\midrule
id & INTEGER & PK, AUTO & 고유 식별자 \\
platform & TEXT & NOT NULL & 플랫폼 \\
platform\_category\_id & TEXT & NOT NULL & 플랫폼 내 카테고리 ID \\
platform\_category\_name & TEXT & NOT NULL & 카테고리 이름 \\
category\_type & TEXT & & GAME $|$ IRL $|$ MUSIC \\
thumbnail\_url & TEXT & & 썸네일 \\
viewer\_count & INTEGER & DEFAULT 0 & 현재 시청자 수 \\
streamer\_count & INTEGER & DEFAULT 0 & 현재 스트리머 수 \\
is\_active & INTEGER & DEFAULT 1 & 활성 여부 \\
first\_seen\_at & DATETIME & DEFAULT NOW & 최초 발견 \\
last\_seen\_at & DATETIME & DEFAULT NOW & 최근 활동 \\
\bottomrule
\multicolumn{4}{l}{\small UNIQUE(platform, platform\_category\_id)} \\
\end{longtable}

\begin{longtable}{L{3.5cm}L{2.5cm}L{2cm}L{5.5cm}}
\caption{unified\_games 테이블 (10 컬럼)} \\
\toprule
\textbf{컬럼} & \textbf{타입} & \textbf{제약} & \textbf{설명} \\
\midrule
id & INTEGER & PK, AUTO & 고유 식별자 \\
name & TEXT & NOT NULL & 게임명 (영문) \\
name\_kr & TEXT & & 게임명 (한국어) \\
genre & TEXT & & 장르 (영문) \\
genre\_kr & TEXT & & 장르 (한국어) \\
developer & TEXT & & 개발사 \\
release\_date & TEXT & & 출시일 \\
description & TEXT & & 설명 \\
image\_url & TEXT & & 대표 이미지 \\
is\_verified & INTEGER & DEFAULT 0 & 검증 여부 \\
\bottomrule
\end{longtable}

\begin{longtable}{L{3.5cm}L{2.5cm}L{2cm}L{5.5cm}}
\caption{category\_game\_mappings 테이블 (7 컬럼)} \\
\toprule
\textbf{컬럼} & \textbf{타입} & \textbf{제약} & \textbf{설명} \\
\midrule
id & INTEGER & PK, AUTO & 고유 식별자 \\
unified\_game\_id & INTEGER & FK, CASCADE & 통합 게임 ID \\
platform & TEXT & NOT NULL & 플랫폼 \\
platform\_category\_id & TEXT & NOT NULL & 플랫폼 카테고리 ID \\
platform\_category\_name & TEXT & & 플랫폼 카테고리 이름 \\
confidence & REAL & DEFAULT 1.0 & 매핑 신뢰도 (0\textasciitilde1) \\
is\_manual & INTEGER & DEFAULT 0 & 수동 매핑 여부 \\
\bottomrule
\multicolumn{4}{l}{\small UNIQUE(platform, platform\_category\_id)} \\
\end{longtable}

% ---- 6.4 오버레이 설정 ----
\section{오버레이 설정 테이블}

오버레이 설정은 \texttt{settings} (글로벌) 및 \texttt{user\_settings} (유저별) 테이블에 JSON 형태로 저장된다. 특수 기능은 전용 테이블을 사용한다.

\begin{table}[H]
\centering
\caption{오버레이 설정 관련 테이블}
\begin{tabularx}{\textwidth}{L{3.5cm}L{2.5cm}X}
\toprule
\textbf{테이블} & \textbf{컬럼 수} & \textbf{용도} \\
\midrule
settings & 2 & 글로벌 설정 (key-value, JSON) \\
user\_settings & 5 & 유저별 설정 (key-value, JSON) \\
roulette\_wheels & 9 & 룰렛 세그먼트, 트리거 조건 \\
signature\_sounds & 9 & 금액별 커스텀 알림음/이미지 \\
emoji\_settings & 8 & 이모지 세트, 애니메이션 스타일 \\
voting\_polls & 7 & 투표 제목, 옵션, 상태 \\
poll\_votes & 5 & 투표 기록 \\
ending\_credits & 10 & 크레딧 섹션, 배경음악, 자동 수집 \\
chat\_bots & 4 & 봇 설정 \\
bot\_commands & 5 & 봇 명령어 \\
\bottomrule
\end{tabularx}
\end{table}

% ---- 6.5 사용자/인증 및 광고/마켓플레이스 ----
\section{사용자/인증 및 광고/마켓플레이스}

\subsection{사용자 시스템}

\begin{longtable}{L{3.5cm}L{2.5cm}L{2cm}L{5.5cm}}
\caption{users 테이블 (12 컬럼)} \\
\toprule
\textbf{컬럼} & \textbf{타입} & \textbf{제약} & \textbf{설명} \\
\midrule
id & INTEGER & PK, AUTO & 고유 식별자 \\
email & TEXT & UNIQUE & 이메일 \\
password\_hash & TEXT & & bcrypt 해시 \\
display\_name & TEXT & NOT NULL & 표시 이름 \\
avatar\_url & TEXT & & 아바타 URL \\
role & TEXT & DEFAULT 'user' & user $|$ creator $|$ advertiser $|$ admin \\
oauth\_provider & TEXT & & OAuth 제공자 \\
oauth\_id & TEXT & & OAuth ID \\
overlay\_hash & TEXT & UNIQUE & 오버레이 고유 해시 (16자 hex) \\
channel\_id & TEXT & & 연결된 채널 ID \\
platform & TEXT & & 주 플랫폼 \\
created\_at & DATETIME & DEFAULT NOW & 가입 시각 \\
\bottomrule
\end{longtable}

\subsection{광고 시스템}

\begin{table}[H]
\centering
\caption{광고 시스템 테이블}
\begin{tabularx}{\textwidth}{L{3cm}C{1.5cm}X}
\toprule
\textbf{테이블} & \textbf{컬럼 수} & \textbf{용도} \\
\midrule
ad\_slots & 9 & 스트리머 광고 슬롯 (위치, 크기, 타입) \\
ad\_campaigns & 15 & 광고주 캠페인 (예산, 기간, 타겟, 상태) \\
ad\_impressions & 6 & 노출/클릭 이벤트 기록 \\
ad\_settlements & 7 & 스트리머 정산 (기간별 노출·클릭·수익) \\
\bottomrule
\end{tabularx}
\end{table}

\subsection{마켓플레이스}

\begin{table}[H]
\centering
\caption{마켓플레이스 테이블}
\begin{tabularx}{\textwidth}{L{3cm}C{1.5cm}X}
\toprule
\textbf{테이블} & \textbf{컬럼 수} & \textbf{용도} \\
\midrule
creators & 8 & 크리에이터 프로필 (다운로드 수, 평점, 인증) \\
designs & 12 & 디자인 템플릿 (카테고리, 승인 워크플로우) \\
design\_reviews & 6 & 리뷰 (1\textasciitilde5점 평점, 텍스트) \\
\bottomrule
\end{tabularx}
\end{table}

% ---- 6.6 실시간 이벤트 구조 ----
\section{실시간 이벤트 구조 (Socket.io)}

\begin{table}[H]
\centering
\caption{Socket.io 이벤트 목록}
\label{tab:socketio}
\begin{tabularx}{\textwidth}{L{3.5cm}L{2.5cm}X}
\toprule
\textbf{이벤트} & \textbf{방향} & \textbf{설명} \\
\midrule
\texttt{join-overlay} & Client $\rightarrow$ Server & 오버레이 룸 참가 (\texttt{overlay:\{hash\}}) \\
\texttt{leave-overlay} & Client $\rightarrow$ Server & 오버레이 룸 퇴장 \\
\texttt{new-event} & Server $\rightarrow$ Client & 채팅/후원/구독/팔로우 이벤트 \\
\texttt{settings-update} & Client $\rightarrow$ Server & 설정 변경 요청 \\
\texttt{settings-updated} & Server $\rightarrow$ Client & 설정 변경 알림 \\
\texttt{roulette-spin} & Bidirectional & 룰렛 스핀 \\
\texttt{emoji-reaction} & Server $\rightarrow$ Client & 단일 이모지 반응 \\
\texttt{emoji-burst} & Server $\rightarrow$ Client & 다중 이모지 반응 \\
\texttt{poll-start/vote/end} & Bidirectional & 투표 라이프사이클 \\
\texttt{credits-start/stop} & Client $\rightarrow$ Server & 엔딩 크레딧 제어 \\
\texttt{ad-slots-updated} & Server $\rightarrow$ Client & 광고 슬롯 변경 알림 \\
\bottomrule
\end{tabularx}
\end{table}

\noindent \texttt{new-event} 페이로드:
\begin{lstlisting}[style=jsstyle]
{
  type: "donation",           // chat | donation | subscribe | follow
  sender: "testuser",         // 발신자 닉네임
  amount: 10000,              // KRW 환산 금액
  message: "test message",    // 메시지
  platform: "soop",           // 플랫폼
  timestamp: "2026-01-30T12:00:00Z",
  id: "uuid-v4"               // 고유 ID
}
\end{lstlisting}


% =============================================================================
% Chapter 7: API 명세
% =============================================================================
\chapter{API 명세}

\section{REST API 엔드포인트}

\begin{table}[H]
\centering
\caption{API 그룹별 엔드포인트 요약}
\begin{tabularx}{\textwidth}{L{3cm}C{1.5cm}X}
\toprule
\textbf{그룹} & \textbf{엔드포인트 수} & \textbf{주요 기능} \\
\midrule
인증 & 6 & 회원가입, 로그인, OAuth, 프로필, 토큰 갱신 \\
설정 & 6 & 글로벌/유저 설정 CRUD, 오버레이 설정 \\
광고 & 10 & 슬롯 CRUD, 캠페인 CRUD, 노출/클릭, 정산 \\
관리자 & 5 & 통계, 스트리머 목록, 플랫폼 통계, 모니터링 \\
모니터링 & 12 & 방송/인물/이벤트/세그먼트/카테고리 조회 \\
카테고리 & 8 & 카테고리 CRUD, 통합 게임, 매핑, 트렌드 \\
플랫폼 & 8 & SOOP/치지직 연결·해제·상태, 채널 정보 \\
헬스체크 & 3 & health, ready, detailed \\
유틸리티 & 2 & 이벤트 시뮬레이션, 이벤트 조회 \\
\midrule
\textbf{합계} & \textbf{$\sim$60} & \\
\bottomrule
\end{tabularx}
\end{table}

\subsection{인증 API}

\begin{table}[H]
\centering
\begin{tabularx}{\textwidth}{L{1.5cm}L{4cm}X}
\toprule
\textbf{Method} & \textbf{Endpoint} & \textbf{설명} \\
\midrule
POST & /api/auth/register & 회원가입 (email, password, displayName) \\
POST & /api/auth/login & 로그인 \\
GET & /api/auth/me & 현재 사용자 정보 \\
PUT & /api/auth/profile & 프로필 업데이트 \\
GET & /api/auth/:provider & OAuth 로그인 (google, naver, twitch, soop) \\
POST & /api/auth/refresh & 토큰 갱신 \\
\bottomrule
\end{tabularx}
\end{table}

\subsection{모니터링 API}

넥슨 내부 대시보드를 위한 핵심 API:

\begin{table}[H]
\centering
\begin{tabularx}{\textwidth}{L{1.5cm}L{5cm}X}
\toprule
\textbf{Method} & \textbf{Endpoint} & \textbf{설명} \\
\midrule
GET & /api/monitor/stats & 전체 통계 (방송·시청자·후원 집계) \\
GET & /api/monitor/stats/timeseries & 시계열 통계 (시간별 추이) \\
GET & /api/monitor/stats/nexon & 넥슨 게임 전용 통계 \\
GET & /api/monitor/broadcasts & 방송 목록 (라이브/종료, 페이지네이션) \\
GET & /api/monitor/persons & 인물 목록 (방송인/시청자, 검색) \\
GET & /api/monitor/persons/:id & 인물 상세 (방송이력, 참여도, 이벤트) \\
GET & /api/monitor/engagement & 시청자 참여도 순위 \\
GET & /api/monitor/events & 이벤트 목록 (타입별 필터) \\
GET & /api/monitor/categories & 카테고리 목록 \\
GET & /api/monitor/segments & 방송 세그먼트 목록 \\
GET & /api/monitor/snapshots & 시청자 스냅샷 목록 \\
GET & /api/monitor/schema & 데이터베이스 스키마 정보 \\
\bottomrule
\end{tabularx}
\end{table}

\section{Socket.io 실시간 이벤트}

Socket.io는 오버레이 실시간 동기화의 핵심이다. 각 오버레이 인스턴스는 \texttt{overlay:\{userHash\}} 룸에 참가하여, 해당 스트리머의 이벤트만 수신한다.

\begin{figure}[H]
\centering
\begin{tikzpicture}[node distance=1cm]
  \node[platform] (platform) {플랫폼\\(SOOP/치지직)};
  \node[adapter, right=2cm of platform] (adapter) {Platform\\Adapter};
  \node[process, right=2cm of adapter] (server) {Express +\\Socket.io};
  \node[client, above right=1cm and 2cm of server] (overlay) {OBS\\오버레이};
  \node[client, below right=1cm and 2cm of server] (dash) {대시보드\\클라이언트};

  \draw[arrow] (platform) -- node[above, font=\scriptsize] {WebSocket} (adapter);
  \draw[arrow] (adapter) -- node[above, font=\scriptsize] {UnifiedEvent} (server);
  \draw[arrow] (server) -- node[above, font=\scriptsize, sloped] {new-event} (overlay);
  \draw[arrow] (server) -- node[below, font=\scriptsize, sloped] {settings-updated} (dash);
  \draw[arrow, dashed] (dash) -- node[right, font=\scriptsize] {settings-update} (server);
\end{tikzpicture}
\caption{Socket.io 이벤트 흐름}
\label{fig:socketio-flow}
\end{figure}


% =============================================================================
% Chapter 12: 개발 로드맵 및 기술 과제
% =============================================================================
\chapter{개발 로드맵 및 기술 과제}

\section{개발 로드맵}

\begin{figure}[H]
\centering
\begin{tikzpicture}[node distance=0.8cm]
  % 세로 타임라인
  \draw[thick, -Stealth] (0,0) -- (0,-14);

  % Phase 마커 (세로)
  \foreach \y/\lab in {0/현재, -4.5/{Phase 1}, -9/{Phase 2}, -13.5/확장} {
    \draw[thick] (-0.3,\y) -- (0.3,\y);
    \node[left=4pt] at (-0.3,\y) {\small\textbf{\lab}};
  }

  % 구현 완료
  \node[box, fill=okgreen!20, draw=okgreen, minimum width=10cm, minimum height=2.8cm, anchor=west] (current) at (1,-2.25) {};
  \node[above=0.05cm, font=\small\bfseries] at (current.north) {구현 완료};
  \node[font=\scriptsize, align=left, text width=9cm] at (current.center) {%
    10종 오버레이 시스템 \quad 인증 (이메일/OAuth) \quad Socket.io 실시간 동기화\\
    광고 관리 시스템 \quad 마켓플레이스 기본 구조 \quad 분석 대시보드\\
    챗봇 시스템 \quad \textbf{35개 DB 테이블}};

  % Phase 1
  \node[box, fill=blue!10, draw=blue!50, minimum width=10cm, minimum height=2.2cm, anchor=west] (phase1) at (1,-6.75) {};
  \node[above=0.05cm, font=\small\bfseries] at (phase1.north) {Phase 1: 스트리머 확보};
  \node[font=\scriptsize, align=left, text width=9cm] at (phase1.center) {%
    넥슨 IP TTS 보이스 \quad 다이내믹 모션 이펙트 \quad 스트리머 성장 대시보드\\
    위플랩 설정 마이그레이션 \quad 게임별 성과 분석 \quad 트위치·유튜브 연동};

  % Phase 2
  \node[box, fill=purple!10, draw=purple!50, minimum width=10cm, minimum height=2.2cm, anchor=west] (phase2) at (1,-11.25) {};
  \node[above=0.05cm, font=\small\bfseries] at (phase2.north) {Phase 2: 넥슨 독점 기능};
  \node[font=\scriptsize, align=left, text width=9cm] at (phase2.center) {%
    게임--방송 실시간 연동 \quad 스트림 드롭스 \quad 인터랙티브 광고\\
    마케팅 기능 활용 \quad 인게임 기능 연동 \quad 시청자 가입유도 전략};

  % 연결선
  \draw[gray, thick] (0,-2.25) -- (current.west);
  \draw[gray, thick] (0,-6.75) -- (phase1.west);
  \draw[gray, thick] (0,-11.25) -- (phase2.west);
\end{tikzpicture}
\caption{개발 로드맵}
\label{fig:roadmap}
\end{figure}

\section{기술 과제}

\begin{table}[H]
\centering
\caption{핵심 기술 과제 및 대응 방안}
\begin{tabularx}{\textwidth}{L{3cm}L{2cm}X}
\toprule
\textbf{과제} & \textbf{우선순위} & \textbf{대응 방안} \\
\midrule
확장성 & 높음 & SQLite에서 PostgreSQL로 마이그레이션, 수평 확장 아키텍처 도입. 이벤트 처리량 10만 건/분 목표 \\
\addlinespace
데이터 파이프라인 고도화 & 높음 & 메시지 큐(Kafka/Redis Streams) 도입, 이벤트 소싱 패턴 적용, 데이터 유실 방지 \\
\addlinespace
실시간 처리 성능 & 중간 & WebSocket 연결 풀 최적화, Socket.io 클러스터링, CDN 기반 오버레이 배포 \\
\addlinespace
데이터 품질 & 중간 & 정규화 파이프라인의 지속적 검증, 이상치 탐지 알고리즘, A/B 테스트 인프라 \\
\addlinespace
보안 강화 & 높음 & API Rate Limiting 고도화, DDoS 방어, 민감 데이터 암호화, 접근 제어 세분화 \\
\addlinespace
모니터링/관측성 & 중간 & 분산 트레이싱, 실시간 알림, 성능 대시보드, 로그 중앙화(ELK/Loki) \\
\bottomrule
\end{tabularx}
\end{table}

\section{추후 확장 플랜}

\begin{enumerate}
  \item \textbf{글로벌 플랫폼 연동}: Twitch EventSub, YouTube Data API v3 완전 통합
  \item \textbf{마이크로서비스 전환}: 모놀리식 Express 서버를 기능별 마이크로서비스로 분리
  \item \textbf{ML 파이프라인}: 시청자 행동 예측, 감성 분석, 이상 탐지 자동화
  \item \textbf{인게임 SDK}: 게임 내 이벤트 $\leftrightarrow$ 방송 오버레이 실시간 연결을 위한 SDK 개발
\end{enumerate}

% =============================================================================
% 결론
% =============================================================================
\chapter*{결론}
\addcontentsline{toc}{chapter}{결론}

본 프로젝트는 단순한 오버레이 툴 개발이 아니라, 스트리밍 생태계의 데이터 주권을 확보하기 위한 전략적 사업이다.

\begin{quote}
\textit{\large ``플랫폼이 데이터를 안 주니까 만든 게 아니라,\\우리가 스트리밍 생태계의 중심이 되기 위해 만든 것이다.''}
\end{quote}

\noindent \textbf{Part~I: 사업 전략}에서 살펴본 바와 같이, Streaming Agent는 스트리밍 데이터를 넥슨의 전략 자산으로 전환하는 데이터 파이프라인이다. 모든 기능을 무료로 제공하여 위플랩 대비 압도적인 가치를 제공하고, 넥슨 IP를 활용한 차별화 콘텐츠로 스트리머 기반을 빠르게 확보한다. 광고 인벤토리를 통한 수익화를 검토 중이나, 서비스 자체의 영리보다 데이터 확보를 통한 넥슨 전체 사업의 의사결정 품질 향상이 핵심 투자 가치다.

\vspace{0.3cm}
\noindent \textbf{Part~II: 기술 명세}에서 제시한 시스템 아키텍처는 이 사업 전략을 뒷받침하는 기술적 기반이다. 플랫폼 어댑터 패턴을 통한 멀티 플랫폼 지원, 35개 테이블로 구성된 정규화 스키마, 그리고 RESTful API + Socket.io 실시간 통신 체계가 안정적인 데이터 수집과 서비스 운영을 보장한다.

\vspace{0.5cm}
\noindent 이 구조의 핵심 장점:

\begin{table}[H]
\centering
\begin{tabularx}{\textwidth}{L{3cm}C{3cm}C{3cm}}
\toprule
\textbf{항목} & \textbf{기존 방식} & \textbf{Streaming Agent} \\
\midrule
플랫폼 허락 & 필수 & 불필요 \\
정책 리스크 & 높음 & 거의 없음 \\
데이터 범위 & API 한계 & 이벤트 무제한 \\
속도 & 느림 (배치) & 실시간 \\
지속 가능성 & 정책 변경 시 중단 & 독립적 \\
수익 모델 & 플랫폼 종속 & 광고 + 데이터 전략 자산 \\
\bottomrule
\end{tabularx}
\end{table}

\begin{center}
\textbf{\large 이건 우회가 아니라 구조적 승리다.}
\end{center}


% =============================================================================
% Appendix A: API 엔드포인트 상세
% =============================================================================
\appendix
\chapter{API 엔드포인트 상세}

\section{인증 API}

\subsection{POST /api/auth/register}

\textbf{Request Body:}
\begin{lstlisting}[style=jsstyle]
{
  "email": "user@example.com",
  "password": "password123",
  "displayName": "nickname"
}
\end{lstlisting}

\textbf{Response (200):}
\begin{lstlisting}[style=jsstyle]
{
  "token": "eyJhbGc...",
  "user": {
    "id": 1,
    "email": "user@example.com",
    "displayName": "nickname",
    "role": "user",
    "overlayHash": "abc123def456"
  }
}
\end{lstlisting}

\section{설정 API}

\subsection{GET /api/settings/:key}

\textbf{Parameters:} \texttt{key} = chat $|$ alert $|$ goal $|$ subtitle $|$ ticker $|$ roulette $|$ emoji $|$ voting $|$ credits

\textbf{Response:}
\begin{lstlisting}[style=jsstyle]
{
  "key": "chat",
  "value": "{\"theme\":\"default\",\"fontSize\":28,...}"
}
\end{lstlisting}

\subsection{POST /api/settings}

\textbf{Request Body:}
\begin{lstlisting}[style=jsstyle]
{
  "key": "chat",
  "value": {
    "theme": "default",
    "fontSize": 28,
    "alignment": "left"
  }
}
\end{lstlisting}

저장 시 Socket.io로 \texttt{settings-updated} 이벤트 브로드캐스트.

\section{모니터링 API}

\subsection{GET /api/monitor/stats}

넥슨 내부 대시보드의 핵심 API:

\begin{lstlisting}[style=jsstyle]
{
  "liveBroadcasts": 150,
  "totalViewers": 45000,
  "totalPersons": 12000,
  "totalDonations": 5600000,
  "platforms": {
    "soop": { "broadcasts": 80, "viewers": 25000 },
    "chzzk": { "broadcasts": 70, "viewers": 20000 }
  },
  "nexon": {
    "soop": { "broadcasts": 15, "viewers": 8000 },
    "chzzk": { "broadcasts": 12, "viewers": 6500 }
  }
}
\end{lstlisting}

\section{에러 응답 형식}

\begin{lstlisting}[style=jsstyle]
{
  "error": "Error message",
  "code": "ERROR_CODE"
}
\end{lstlisting}

\begin{table}[H]
\centering
\begin{tabular}{ll}
\toprule
\textbf{코드} & \textbf{설명} \\
\midrule
400 & Bad Request --- 잘못된 요청 \\
401 & Unauthorized --- 인증 필요 \\
403 & Forbidden --- 권한 없음 \\
404 & Not Found --- 리소스 없음 \\
500 & Internal Server Error \\
\bottomrule
\end{tabular}
\end{table}


% =============================================================================
% Appendix B: CREATE TABLE 전문
% =============================================================================
\chapter{데이터베이스 CREATE TABLE 전문}

\section{코어 스트리밍}

\begin{lstlisting}[style=sqlstyle, caption={persons}]
CREATE TABLE persons (
  id INTEGER PRIMARY KEY AUTOINCREMENT,
  platform TEXT NOT NULL CHECK(platform IN ('soop','chzzk','twitch','youtube')),
  platform_user_id TEXT NOT NULL,
  nickname TEXT,
  profile_image_url TEXT,
  channel_id TEXT,
  channel_description TEXT,
  follower_count INTEGER DEFAULT 0,
  subscriber_count INTEGER DEFAULT 0,
  total_broadcast_minutes INTEGER DEFAULT 0,
  last_broadcast_at DATETIME,
  first_seen_at DATETIME DEFAULT CURRENT_TIMESTAMP,
  last_seen_at DATETIME DEFAULT CURRENT_TIMESTAMP,
  created_at DATETIME DEFAULT CURRENT_TIMESTAMP,
  updated_at DATETIME DEFAULT CURRENT_TIMESTAMP,
  UNIQUE(platform, platform_user_id)
);
\end{lstlisting}

\begin{lstlisting}[style=sqlstyle, caption={events}]
CREATE TABLE events (
  id TEXT PRIMARY KEY,
  event_type TEXT NOT NULL CHECK(event_type IN
    ('chat','donation','subscribe','follow','view')),
  platform TEXT NOT NULL CHECK(platform IN ('soop','chzzk','twitch','youtube')),
  actor_person_id INTEGER REFERENCES persons(id),
  actor_nickname TEXT,
  actor_role TEXT CHECK(actor_role IN
    ('streamer','manager','vip','fan','regular','system')),
  target_person_id INTEGER REFERENCES persons(id),
  target_channel_id TEXT NOT NULL,
  broadcast_id INTEGER REFERENCES broadcasts(id),
  message TEXT,
  amount INTEGER,
  original_amount INTEGER,
  currency TEXT,
  donation_type TEXT,
  event_timestamp DATETIME NOT NULL,
  ingested_at DATETIME DEFAULT CURRENT_TIMESTAMP
);
\end{lstlisting}

\begin{lstlisting}[style=sqlstyle, caption={broadcasts}]
CREATE TABLE broadcasts (
  id INTEGER PRIMARY KEY AUTOINCREMENT,
  platform TEXT NOT NULL CHECK(platform IN ('soop','chzzk','twitch','youtube')),
  channel_id TEXT NOT NULL,
  broadcast_id TEXT NOT NULL,
  broadcaster_person_id INTEGER REFERENCES persons(id),
  title TEXT,
  thumbnail_url TEXT,
  current_viewer_count INTEGER DEFAULT 0,
  peak_viewer_count INTEGER DEFAULT 0,
  avg_viewer_count INTEGER DEFAULT 0,
  viewer_sum INTEGER DEFAULT 0,
  snapshot_count INTEGER DEFAULT 0,
  is_live INTEGER DEFAULT 1,
  started_at DATETIME,
  ended_at DATETIME,
  duration_minutes INTEGER,
  recorded_at DATETIME DEFAULT CURRENT_TIMESTAMP,
  updated_at DATETIME DEFAULT CURRENT_TIMESTAMP,
  UNIQUE(platform, channel_id, broadcast_id)
);
\end{lstlisting}

\section{사용자 시스템}

\begin{lstlisting}[style=sqlstyle, caption={users}]
CREATE TABLE users (
  id INTEGER PRIMARY KEY AUTOINCREMENT,
  email TEXT UNIQUE,
  password_hash TEXT,
  display_name TEXT NOT NULL,
  avatar_url TEXT,
  role TEXT DEFAULT 'user' CHECK(role IN ('user','creator','advertiser','admin')),
  oauth_provider TEXT,
  oauth_id TEXT,
  overlay_hash TEXT UNIQUE,
  channel_id TEXT,
  platform TEXT,
  created_at DATETIME DEFAULT CURRENT_TIMESTAMP
);
\end{lstlisting}


% =============================================================================
% Appendix C: 플랫폼별 이벤트 정규화 매핑
% =============================================================================
\chapter{플랫폼별 이벤트 정규화 매핑}

\begin{longtable}{L{2cm}L{3cm}L{3cm}L{2.5cm}L{3cm}}
\caption{플랫폼 원본 $\rightarrow$ UnifiedEvent 매핑} \\
\toprule
\textbf{플랫폼} & \textbf{원본 필드} & \textbf{UnifiedEvent} & \textbf{변환} & \textbf{비고} \\
\midrule
\endfirsthead
\toprule
\textbf{플랫폼} & \textbf{원본 필드} & \textbf{UnifiedEvent} & \textbf{변환} & \textbf{비고} \\
\midrule
\endhead

SOOP & bj\_id & sender.id & 직접 매핑 & \\
SOOP & user\_nick & sender.nickname & 직접 매핑 & \\
SOOP & star\_cnt & content.amount & $\times$ 100 KRW & 별풍선 \\
SOOP & ad\_amount & content.amount & 직접 KRW & 애드벌룬 \\
SOOP & grade & sender.role & 등급 변환 & 1=streamer 등 \\
\midrule

치지직 & uid & sender.id & 직접 매핑 & \\
치지직 & nickname & sender.nickname & 직접 매핑 & \\
치지직 & payAmount & content.amount & 직접 KRW & 치즈 \\
치지직 & profile.userRoleCode & sender.role & 코드 변환 & streamer/manager 등 \\
\midrule

YouTube & authorChannelId & sender.id & 직접 매핑 & \\
YouTube & displayName & sender.nickname & 직접 매핑 & \\
YouTube & amountMicros & content.amount & USD$\rightarrow$KRW & Super Chat \\
YouTube & currency & content.currency & 보존 & 다국가 통화 \\
\midrule

Twitch & user\_id & sender.id & 직접 매핑 & \\
Twitch & user\_name & sender.nickname & 직접 매핑 & \\
Twitch & bits & content.amount & $\times$ 14 KRW & Bits \\
Twitch & badge\_info & sender.role & 배지 변환 & subscriber/vip 등 \\
\bottomrule
\end{longtable}

\end{document}